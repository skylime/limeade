% Generated by Sphinx.
\def\sphinxdocclass{report}
\documentclass[letterpaper,10pt,english]{sphinxmanual}
\usepackage[utf8]{inputenc}
\DeclareUnicodeCharacter{00A0}{\nobreakspace}
\usepackage[T1]{fontenc}
\usepackage{babel}
\usepackage{times}
\usepackage[Bjarne]{fncychap}
\usepackage{longtable}
\usepackage{sphinx}
\usepackage{multirow}


\title{limeade Documentation}
\date{September 09, 2012}
\release{1.0}
\author{Marc Rochow}
\newcommand{\sphinxlogo}{}
\renewcommand{\releasename}{Release}
\makeindex

\makeatletter
\def\PYG@reset{\let\PYG@it=\relax \let\PYG@bf=\relax%
    \let\PYG@ul=\relax \let\PYG@tc=\relax%
    \let\PYG@bc=\relax \let\PYG@ff=\relax}
\def\PYG@tok#1{\csname PYG@tok@#1\endcsname}
\def\PYG@toks#1+{\ifx\relax#1\empty\else%
    \PYG@tok{#1}\expandafter\PYG@toks\fi}
\def\PYG@do#1{\PYG@bc{\PYG@tc{\PYG@ul{%
    \PYG@it{\PYG@bf{\PYG@ff{#1}}}}}}}
\def\PYG#1#2{\PYG@reset\PYG@toks#1+\relax+\PYG@do{#2}}

\expandafter\def\csname PYG@tok@gd\endcsname{\def\PYG@tc##1{\textcolor[rgb]{0.63,0.00,0.00}{##1}}}
\expandafter\def\csname PYG@tok@gu\endcsname{\let\PYG@bf=\textbf\def\PYG@tc##1{\textcolor[rgb]{0.50,0.00,0.50}{##1}}}
\expandafter\def\csname PYG@tok@gt\endcsname{\def\PYG@tc##1{\textcolor[rgb]{0.00,0.25,0.82}{##1}}}
\expandafter\def\csname PYG@tok@gs\endcsname{\let\PYG@bf=\textbf}
\expandafter\def\csname PYG@tok@gr\endcsname{\def\PYG@tc##1{\textcolor[rgb]{1.00,0.00,0.00}{##1}}}
\expandafter\def\csname PYG@tok@cm\endcsname{\let\PYG@it=\textit\def\PYG@tc##1{\textcolor[rgb]{0.25,0.50,0.56}{##1}}}
\expandafter\def\csname PYG@tok@vg\endcsname{\def\PYG@tc##1{\textcolor[rgb]{0.73,0.38,0.84}{##1}}}
\expandafter\def\csname PYG@tok@m\endcsname{\def\PYG@tc##1{\textcolor[rgb]{0.13,0.50,0.31}{##1}}}
\expandafter\def\csname PYG@tok@mh\endcsname{\def\PYG@tc##1{\textcolor[rgb]{0.13,0.50,0.31}{##1}}}
\expandafter\def\csname PYG@tok@cs\endcsname{\def\PYG@tc##1{\textcolor[rgb]{0.25,0.50,0.56}{##1}}\def\PYG@bc##1{\setlength{\fboxsep}{0pt}\colorbox[rgb]{1.00,0.94,0.94}{\strut ##1}}}
\expandafter\def\csname PYG@tok@ge\endcsname{\let\PYG@it=\textit}
\expandafter\def\csname PYG@tok@vc\endcsname{\def\PYG@tc##1{\textcolor[rgb]{0.73,0.38,0.84}{##1}}}
\expandafter\def\csname PYG@tok@il\endcsname{\def\PYG@tc##1{\textcolor[rgb]{0.13,0.50,0.31}{##1}}}
\expandafter\def\csname PYG@tok@go\endcsname{\def\PYG@tc##1{\textcolor[rgb]{0.19,0.19,0.19}{##1}}}
\expandafter\def\csname PYG@tok@cp\endcsname{\def\PYG@tc##1{\textcolor[rgb]{0.00,0.44,0.13}{##1}}}
\expandafter\def\csname PYG@tok@gi\endcsname{\def\PYG@tc##1{\textcolor[rgb]{0.00,0.63,0.00}{##1}}}
\expandafter\def\csname PYG@tok@gh\endcsname{\let\PYG@bf=\textbf\def\PYG@tc##1{\textcolor[rgb]{0.00,0.00,0.50}{##1}}}
\expandafter\def\csname PYG@tok@ni\endcsname{\let\PYG@bf=\textbf\def\PYG@tc##1{\textcolor[rgb]{0.84,0.33,0.22}{##1}}}
\expandafter\def\csname PYG@tok@nl\endcsname{\let\PYG@bf=\textbf\def\PYG@tc##1{\textcolor[rgb]{0.00,0.13,0.44}{##1}}}
\expandafter\def\csname PYG@tok@nn\endcsname{\let\PYG@bf=\textbf\def\PYG@tc##1{\textcolor[rgb]{0.05,0.52,0.71}{##1}}}
\expandafter\def\csname PYG@tok@no\endcsname{\def\PYG@tc##1{\textcolor[rgb]{0.38,0.68,0.84}{##1}}}
\expandafter\def\csname PYG@tok@na\endcsname{\def\PYG@tc##1{\textcolor[rgb]{0.25,0.44,0.63}{##1}}}
\expandafter\def\csname PYG@tok@nb\endcsname{\def\PYG@tc##1{\textcolor[rgb]{0.00,0.44,0.13}{##1}}}
\expandafter\def\csname PYG@tok@nc\endcsname{\let\PYG@bf=\textbf\def\PYG@tc##1{\textcolor[rgb]{0.05,0.52,0.71}{##1}}}
\expandafter\def\csname PYG@tok@nd\endcsname{\let\PYG@bf=\textbf\def\PYG@tc##1{\textcolor[rgb]{0.33,0.33,0.33}{##1}}}
\expandafter\def\csname PYG@tok@ne\endcsname{\def\PYG@tc##1{\textcolor[rgb]{0.00,0.44,0.13}{##1}}}
\expandafter\def\csname PYG@tok@nf\endcsname{\def\PYG@tc##1{\textcolor[rgb]{0.02,0.16,0.49}{##1}}}
\expandafter\def\csname PYG@tok@si\endcsname{\let\PYG@it=\textit\def\PYG@tc##1{\textcolor[rgb]{0.44,0.63,0.82}{##1}}}
\expandafter\def\csname PYG@tok@s2\endcsname{\def\PYG@tc##1{\textcolor[rgb]{0.25,0.44,0.63}{##1}}}
\expandafter\def\csname PYG@tok@vi\endcsname{\def\PYG@tc##1{\textcolor[rgb]{0.73,0.38,0.84}{##1}}}
\expandafter\def\csname PYG@tok@nt\endcsname{\let\PYG@bf=\textbf\def\PYG@tc##1{\textcolor[rgb]{0.02,0.16,0.45}{##1}}}
\expandafter\def\csname PYG@tok@nv\endcsname{\def\PYG@tc##1{\textcolor[rgb]{0.73,0.38,0.84}{##1}}}
\expandafter\def\csname PYG@tok@s1\endcsname{\def\PYG@tc##1{\textcolor[rgb]{0.25,0.44,0.63}{##1}}}
\expandafter\def\csname PYG@tok@gp\endcsname{\let\PYG@bf=\textbf\def\PYG@tc##1{\textcolor[rgb]{0.78,0.36,0.04}{##1}}}
\expandafter\def\csname PYG@tok@sh\endcsname{\def\PYG@tc##1{\textcolor[rgb]{0.25,0.44,0.63}{##1}}}
\expandafter\def\csname PYG@tok@ow\endcsname{\let\PYG@bf=\textbf\def\PYG@tc##1{\textcolor[rgb]{0.00,0.44,0.13}{##1}}}
\expandafter\def\csname PYG@tok@sx\endcsname{\def\PYG@tc##1{\textcolor[rgb]{0.78,0.36,0.04}{##1}}}
\expandafter\def\csname PYG@tok@bp\endcsname{\def\PYG@tc##1{\textcolor[rgb]{0.00,0.44,0.13}{##1}}}
\expandafter\def\csname PYG@tok@c1\endcsname{\let\PYG@it=\textit\def\PYG@tc##1{\textcolor[rgb]{0.25,0.50,0.56}{##1}}}
\expandafter\def\csname PYG@tok@kc\endcsname{\let\PYG@bf=\textbf\def\PYG@tc##1{\textcolor[rgb]{0.00,0.44,0.13}{##1}}}
\expandafter\def\csname PYG@tok@c\endcsname{\let\PYG@it=\textit\def\PYG@tc##1{\textcolor[rgb]{0.25,0.50,0.56}{##1}}}
\expandafter\def\csname PYG@tok@mf\endcsname{\def\PYG@tc##1{\textcolor[rgb]{0.13,0.50,0.31}{##1}}}
\expandafter\def\csname PYG@tok@err\endcsname{\def\PYG@bc##1{\setlength{\fboxsep}{0pt}\fcolorbox[rgb]{1.00,0.00,0.00}{1,1,1}{\strut ##1}}}
\expandafter\def\csname PYG@tok@kd\endcsname{\let\PYG@bf=\textbf\def\PYG@tc##1{\textcolor[rgb]{0.00,0.44,0.13}{##1}}}
\expandafter\def\csname PYG@tok@ss\endcsname{\def\PYG@tc##1{\textcolor[rgb]{0.32,0.47,0.09}{##1}}}
\expandafter\def\csname PYG@tok@sr\endcsname{\def\PYG@tc##1{\textcolor[rgb]{0.14,0.33,0.53}{##1}}}
\expandafter\def\csname PYG@tok@mo\endcsname{\def\PYG@tc##1{\textcolor[rgb]{0.13,0.50,0.31}{##1}}}
\expandafter\def\csname PYG@tok@mi\endcsname{\def\PYG@tc##1{\textcolor[rgb]{0.13,0.50,0.31}{##1}}}
\expandafter\def\csname PYG@tok@kn\endcsname{\let\PYG@bf=\textbf\def\PYG@tc##1{\textcolor[rgb]{0.00,0.44,0.13}{##1}}}
\expandafter\def\csname PYG@tok@o\endcsname{\def\PYG@tc##1{\textcolor[rgb]{0.40,0.40,0.40}{##1}}}
\expandafter\def\csname PYG@tok@kr\endcsname{\let\PYG@bf=\textbf\def\PYG@tc##1{\textcolor[rgb]{0.00,0.44,0.13}{##1}}}
\expandafter\def\csname PYG@tok@s\endcsname{\def\PYG@tc##1{\textcolor[rgb]{0.25,0.44,0.63}{##1}}}
\expandafter\def\csname PYG@tok@kp\endcsname{\def\PYG@tc##1{\textcolor[rgb]{0.00,0.44,0.13}{##1}}}
\expandafter\def\csname PYG@tok@w\endcsname{\def\PYG@tc##1{\textcolor[rgb]{0.73,0.73,0.73}{##1}}}
\expandafter\def\csname PYG@tok@kt\endcsname{\def\PYG@tc##1{\textcolor[rgb]{0.56,0.13,0.00}{##1}}}
\expandafter\def\csname PYG@tok@sc\endcsname{\def\PYG@tc##1{\textcolor[rgb]{0.25,0.44,0.63}{##1}}}
\expandafter\def\csname PYG@tok@sb\endcsname{\def\PYG@tc##1{\textcolor[rgb]{0.25,0.44,0.63}{##1}}}
\expandafter\def\csname PYG@tok@k\endcsname{\let\PYG@bf=\textbf\def\PYG@tc##1{\textcolor[rgb]{0.00,0.44,0.13}{##1}}}
\expandafter\def\csname PYG@tok@se\endcsname{\let\PYG@bf=\textbf\def\PYG@tc##1{\textcolor[rgb]{0.25,0.44,0.63}{##1}}}
\expandafter\def\csname PYG@tok@sd\endcsname{\let\PYG@it=\textit\def\PYG@tc##1{\textcolor[rgb]{0.25,0.44,0.63}{##1}}}

\def\PYGZbs{\char`\\}
\def\PYGZus{\char`\_}
\def\PYGZob{\char`\{}
\def\PYGZcb{\char`\}}
\def\PYGZca{\char`\^}
\def\PYGZam{\char`\&}
\def\PYGZlt{\char`\<}
\def\PYGZgt{\char`\>}
\def\PYGZsh{\char`\#}
\def\PYGZpc{\char`\%}
\def\PYGZdl{\char`\$}
\def\PYGZti{\char`\~}
% for compatibility with earlier versions
\def\PYGZat{@}
\def\PYGZlb{[}
\def\PYGZrb{]}
\makeatother

\begin{document}

\maketitle
\tableofcontents
\phantomsection\label{index::doc}


Willkommen zur Version 1.0 der limeade Dokumentation.


\chapter{Einführung}
\label{index:limeade-dokumentation}\label{index:einfuhrung}

\section{Überblick}
\label{getting-started/overview:uberblick}\label{getting-started/overview::doc}
limeade ist ein Projekt der SkyLime GbR und wurde von Marc Rochow als
Bachelorarbeit bei der Hochschule Augsburg weiterentwickelt.

Das Webinterface stellt eine voll funktionstüchtige Webanwendung dar, die
speziell für Webhosting Firmen gedacht ist. Es soll das Verwalten und
Administrieren beschleunigen und vereinfachen.


\subsection{Features}
\label{getting-started/overview:features}
Verwalten und Anlegen von
\begin{itemize}
\item {} 
VHosts

\item {} 
FTP

\item {} 
MySQL Datenbanken

\item {} 
Backups

\item {} 
Cloud Instanzen

\item {} 
Domains

\item {} 
SSL Zertifikaten

\item {} 
E-mail Adressen, Weiterleitungen und Mailbboxen

\item {} 
VNC

\end{itemize}


\section{Requirements}
\label{getting-started/requirements:requirements}\label{getting-started/requirements::doc}\begin{itemize}
\item {} 
\href{http://www.python.org/}{Python} == 2.7

\item {} 
\href{http://www.djangoproject.com/}{Django} \textgreater{}= 1.3

\item {} 
\href{http://www.nodejs.org}{Node.js} \textgreater{}= 0.6

\end{itemize}


\subsection{weitere Anforderungen}
\label{getting-started/requirements:weitere-anforderungen}\label{getting-started/requirements:node-js}\begin{itemize}
\item {} 
\href{http://celeryproject.org/}{Celery} und \href{http://docs.celeryproject.org/en/latest/django/index.html}{django-celery}

\item {} 
\href{http://packages.python.org/pyOpenSSL/}{pyOpenSSL}

\item {} 
\href{http://lxml.de/}{lxml}

\item {} 
\href{http://c0re.23.nu/c0de/IPy/}{Ipy}

\item {} 
\href{http://libvirt.org/}{libvirt}

\item {} 
\href{http://www.rabbitmq.com/}{RabbitMQ}

\end{itemize}


\subsection{Integrierte Anwendungen (Django)}
\label{getting-started/requirements:integrierte-anwendungen-django}\label{getting-started/requirements:rabbitmq}\begin{itemize}
\item {} 
\href{http://south.aeracode.org/}{South}

\item {} 
\href{http://django-uni-form.rtfd.org/}{django-uni-form}

\end{itemize}


\section{Installation}
\label{getting-started/install:django-uni-form}\label{getting-started/install:installation}\label{getting-started/install::doc}
Der Source Code kann von Github geladen werden. Entweder per direkten Download
oder über Git.

\begin{Verbatim}[commandchars=\\\{\}]
git clone git@github.com:fatality/limeade.git
\end{Verbatim}


\subsection{Installation der Anforderungen}
\label{getting-started/install:installation-der-anforderungen}
Die Installation der Anforderungen wird beispielhaft für \textbf{Arch Linux}
beschrieben.

\textbf{Celery und django-celery:}

\begin{Verbatim}[commandchars=\\\{\}]
pip2 install -U Celery
pip2 install -U django-celery
\end{Verbatim}

\textbf{pyOpenSSL:}

\begin{Verbatim}[commandchars=\\\{\}]
pacman -S python2-pyopenssl
\end{Verbatim}

\textbf{lxml:}

\begin{Verbatim}[commandchars=\\\{\}]
pacman -S python2-lxml
\end{Verbatim}

\textbf{Ipy:}

\begin{Verbatim}[commandchars=\\\{\}]
pacman -S python2-ipy
\end{Verbatim}

\textbf{libvirt:}

Um libvirt und somit auch KVM benutzen zu können, muss der Computer
Virtualisierung unterstützen. Dies lässt sich mit folgendem Befehl testen:
\DUspan{s2}{}\DUspan{o}{}
\begin{Verbatim}[commandchars=\\\{\}]
grep -E "(vmx\textbar{}svm)" --color=always /proc/cpuinfo
\end{Verbatim}

Wenn die Ausgabe korrekt ist und der Computer Virtualisierung unterstützt kann
libvirt, KVM und QEMU installiert und eingerichtet werden.

\begin{Verbatim}[commandchars=\\\{\}]
pacman -S qemu-kvm libvirt dnsmasq virt-manager
\end{Verbatim}

Um dnsmasq korrekt einzurichten empfiehlt sich folgende diese \href{https://wiki.archlinux.org/index.php/Dnsmasq}{Anleitung}.
Libvirt als Normaluser verwenden zu können ist in dieser \href{https://wiki.archlinux.org/index.php/Libvirt\#Configuration}{Anleitung}
beschrieben.

\textbf{libvirt und TCP:}

Die Datei \emph{/etc/libvirt/libvirtd.conf} öffnen und folgende Stellen ändern:

\begin{Verbatim}[commandchars=\\\{\}]
\PYG{n}{listen\PYGZus{}tls} \PYG{o}{=} \PYG{l+m+mi}{0}
\PYG{n}{listen\PYGZus{}tcp} \PYG{o}{=} \PYG{l+m+mi}{1}
\PYG{n}{auth\PYGZus{}tcp}\PYG{o}{=}\PYG{n}{none}
\end{Verbatim}

Die Deamon Datei \emph{/etc/conf.d/libvirtd} öffnen und den Eintrag
\emph{LIBVIRTD\_ARGS} in \emph{LIBVIRTD\_ARGS=''--listen''} ändern.

Als letzter Schritt die QEMU Konfiguration in libvirt (\emph{/etc/libvirt/qemu.conf})
öffnen und \emph{vnc\_listen = ``0.0.0.0''} eintragen bzw. den Kommentar entfernen.

Als nächstes kann mittels virt-manager eine VM angelegt werden. Die Daten der VM
können im Django Admin später eingetragen werden.

\textbf{RabbitMQ:}

In den \href{https://aur.archlinux.org/}{Arch User Repositories} findet sich ein \href{http://aur.archlinux.org/packages.php?ID=19090}{Paket für RabbitMQ} welches
installiert werden muss.

Anschließend wird der RabbitMQ als root gestartet:

\begin{Verbatim}[commandchars=\\\{\}]
\PYG{n}{rabbitmq}\PYG{o}{-}\PYG{n}{server}
\end{Verbatim}

Läuft der Server, müssen folgende drei Schritte durchgeführt werden:
\DUspan{s2}{}\DUspan{s2}{}\DUspan{s2}{}
\begin{Verbatim}[commandchars=\\\{\}]
rabbitmqctl add\_user limeade EimequuChuap8aa8ohyo
rabbitmqctl add\_vhost limeade
rabbitmqctl set\_permissions -p limeade limeade ".*" ".*" ".*"
\end{Verbatim}

Der Server läuft nun und empfängt Nachrichten, leitet diese aber noch nicht
weiter. Dazu muss der Deamon limed mittels Celery gestartet werden. In der
\emph{settings.py} müssen dazu noch die Angaben für MySQL gemacht werden, damit
völlig automatisch Datenbanken erstellt werden können.


\section{Benutzung}
\label{getting-started/using::doc}\label{getting-started/using:benutzung}
Sollten Node.js, Python, Django und die anderen Abhängigkeiten installiert sein
lässt sich die Webanwendung mit Django einrichten. Dazu sollte jedoch eine
lokale Konfigurationsdatei erstellt werden. Eine Beispieldatei ist integriert
(local\_settings.py.example).


\subsection{Django}
\label{getting-started/using:django}
Minimal sollte eine Datenbank angelegt werden und folgender Schritt durchgeführt
werden:
\DUspan{nb}{}
\begin{Verbatim}[commandchars=\\\{\}]
cd web/limeade
python2 manage.py syncdb --migrate
\end{Verbatim}

Dies erstellt alle Tabellen in der Datenbank und zugleich auch einen Benutzer
mit vollen Adminrechten. Starten lässt sich die Anwendung anschließend lokal mit

\begin{Verbatim}[commandchars=\\\{\}]
python2 manage.py runserver
\end{Verbatim}

Die Webanwendung läuft nun unter \href{http://127.0.0.1:8000/system/}{http://127.0.0.1:8000/system/} und kann im Admin
mit Daten gefüttert werden (\href{http://127.0.0.1:8000/admin/}{http://127.0.0.1:8000/admin/}).


\subsection{Node.js Proxy}
\label{getting-started/using:id2}\label{getting-started/using:node-js-proxy}
Der Node.js Proxy wird im \emph{proxy}-Verzeichnis ebenfalls ausgeführt:

\begin{Verbatim}[commandchars=\\\{\}]
node index.js
\end{Verbatim}


\subsection{RabbitMQ und Celery}
\label{getting-started/using:rabbitmq-und-celery}
Zum Abschluss muss RabbitMQ und der Deamon im \emph{limed}-Verzeichnis gestaret werden:

\begin{Verbatim}[commandchars=\\\{\}]
\PYG{n}{rabbitmq}\PYG{o}{-}\PYG{n}{server}
\PYG{n}{celeryd}
\end{Verbatim}


\chapter{API Dokumentation}
\label{index:api-dokumentation}

\section{System}
\label{api/system::doc}\label{api/system:system}

\subsection{Models}
\label{api/system:models}\label{api/system:module-limeade.system.models}\index{limeade.system.models (module)}
Models for limeade system
\index{Contract (class in limeade.system.models)}

\begin{fulllineitems}
\phantomsection\label{api/system:limeade.system.models.Contract}\pysiglinewithargsret{\strong{class }\code{limeade.system.models.}\bfcode{Contract}}{\emph{*args}, \emph{**kwargs}}{}
Creates a contract for a person with a product.
\begin{quote}\begin{description}
\item[{Parameters}] \leavevmode\begin{itemize}
\item {} 
\textbf{person} -- the person

\item {} 
\textbf{product} -- the product

\end{itemize}

\item[{Example }] \leavevmode
\begin{Verbatim}[commandchars=\\\{\}]
\PYG{g+gp}{\PYGZgt{}\PYGZgt{}\PYGZgt{} }\PYG{k+kn}{from} \PYG{n+nn}{limeade.system.models} \PYG{k+kn}{import} \PYG{n}{Contract}
\PYG{g+gp}{\PYGZgt{}\PYGZgt{}\PYGZgt{} }\PYG{n}{contract} \PYG{o}{=} \PYG{n}{Contract}\PYG{o}{.}\PYG{n}{objects}\PYG{o}{.}\PYG{n}{get}\PYG{p}{(}\PYG{n}{pk}\PYG{o}{=}\PYG{l+m+mi}{1}\PYG{p}{)}
\PYG{g+gp}{\PYGZgt{}\PYGZgt{}\PYGZgt{} }\PYG{n}{contract}
\PYG{g+go}{\PYGZlt{}Contract: Testuser / Testproduct\PYGZgt{}}
\end{Verbatim}

\end{description}\end{quote}
\index{has\_add\_permission() (limeade.system.models.Contract method)}

\begin{fulllineitems}
\phantomsection\label{api/system:limeade.system.models.Contract.has_add_permission}\pysiglinewithargsret{\bfcode{has\_add\_permission}}{\emph{request}}{}
Returns true if user has permission to add a contract.

\end{fulllineitems}

\index{has\_change\_permission() (limeade.system.models.Contract method)}

\begin{fulllineitems}
\phantomsection\label{api/system:limeade.system.models.Contract.has_change_permission}\pysiglinewithargsret{\bfcode{has\_change\_permission}}{\emph{request}, \emph{obj}}{}
Returns true if user has permission to change the contract.

\end{fulllineitems}

\index{has\_delete\_permission() (limeade.system.models.Contract method)}

\begin{fulllineitems}
\phantomsection\label{api/system:limeade.system.models.Contract.has_delete_permission}\pysiglinewithargsret{\bfcode{has\_delete\_permission}}{\emph{request}, \emph{obj}}{}
Returns true if user has permission to delete the contract.

\end{fulllineitems}

\index{queryset() (limeade.system.models.Contract method)}

\begin{fulllineitems}
\phantomsection\label{api/system:limeade.system.models.Contract.queryset}\pysiglinewithargsret{\bfcode{queryset}}{\emph{request}}{}
Returns a filtered queryset by request.

\end{fulllineitems}


\end{fulllineitems}

\index{Domain (class in limeade.system.models)}

\begin{fulllineitems}
\phantomsection\label{api/system:limeade.system.models.Domain}\pysiglinewithargsret{\strong{class }\code{limeade.system.models.}\bfcode{Domain}}{\emph{*args}, \emph{**kwargs}}{}
Model for a domain, which belongs to a contract.
\begin{quote}\begin{description}
\item[{Parameters}] \leavevmode\begin{itemize}
\item {} 
\textbf{contract} -- the contract

\item {} 
\textbf{name} -- the name of the domain

\end{itemize}

\item[{Example }] \leavevmode
\begin{Verbatim}[commandchars=\\\{\}]
\PYG{g+gp}{\PYGZgt{}\PYGZgt{}\PYGZgt{} }\PYG{k+kn}{from} \PYG{n+nn}{limeade.system.models} \PYG{k+kn}{import} \PYG{n}{Domain}
\PYG{g+gp}{\PYGZgt{}\PYGZgt{}\PYGZgt{} }\PYG{n}{domain} \PYG{o}{=} \PYG{n}{Domain}\PYG{o}{.}\PYG{n}{objects}\PYG{o}{.}\PYG{n}{get}\PYG{p}{(}\PYG{n}{pk}\PYG{o}{=}\PYG{l+m+mi}{1}\PYG{p}{)}
\PYG{g+gp}{\PYGZgt{}\PYGZgt{}\PYGZgt{} }\PYG{n}{domain}
\PYG{g+go}{\PYGZlt{}Domain: testdomain.com\PYGZgt{}}
\end{Verbatim}

\end{description}\end{quote}

\begin{notice}{note}{Note:}
name of domain must be unique
\end{notice}
\index{owner() (limeade.system.models.Domain method)}

\begin{fulllineitems}
\phantomsection\label{api/system:limeade.system.models.Domain.owner}\pysiglinewithargsret{\bfcode{owner}}{}{}
Returns the owner of the domain.

\end{fulllineitems}


\end{fulllineitems}

\index{Person (class in limeade.system.models)}

\begin{fulllineitems}
\phantomsection\label{api/system:limeade.system.models.Person}\pysiglinewithargsret{\strong{class }\code{limeade.system.models.}\bfcode{Person}}{\emph{*args}, \emph{**kwargs}}{}
Creates a user profile for any registered person.
\begin{quote}\begin{description}
\item[{Parameters}] \leavevmode\begin{itemize}
\item {} 
\textbf{user} -- the user stored in django.auth.models.User

\item {} 
\textbf{company} -- if available, the company the user belongs to

\item {} 
\textbf{address} -- default address of user

\item {} 
\textbf{parent} -- persons can have own customers (e.g. reseller \textgreater{} customer)

\end{itemize}

\item[{Example }] \leavevmode
\begin{Verbatim}[commandchars=\\\{\}]
\PYG{g+gp}{\PYGZgt{}\PYGZgt{}\PYGZgt{} }\PYG{k+kn}{from} \PYG{n+nn}{limeade.system.models} \PYG{k+kn}{import} \PYG{n}{Person}
\PYG{g+gp}{\PYGZgt{}\PYGZgt{}\PYGZgt{} }\PYG{n}{person} \PYG{o}{=} \PYG{n}{Person}\PYG{o}{.}\PYG{n}{objects}\PYG{o}{.}\PYG{n}{get}\PYG{p}{(}\PYG{n}{pk}\PYG{o}{=}\PYG{l+m+mi}{1}\PYG{p}{)}
\PYG{g+gp}{\PYGZgt{}\PYGZgt{}\PYGZgt{} }\PYG{n}{person}
\PYG{g+go}{\PYGZlt{}Person: Test User (Test Company)\PYGZgt{}}
\end{Verbatim}

\end{description}\end{quote}
\index{first\_name() (limeade.system.models.Person method)}

\begin{fulllineitems}
\phantomsection\label{api/system:limeade.system.models.Person.first_name}\pysiglinewithargsret{\bfcode{first\_name}}{}{}
Returns the first name of the user.

\end{fulllineitems}

\index{last\_name() (limeade.system.models.Person method)}

\begin{fulllineitems}
\phantomsection\label{api/system:limeade.system.models.Person.last_name}\pysiglinewithargsret{\bfcode{last\_name}}{}{}
Returns the last name of the user.

\end{fulllineitems}

\index{system\_user\_home() (limeade.system.models.Person method)}

\begin{fulllineitems}
\phantomsection\label{api/system:limeade.system.models.Person.system_user_home}\pysiglinewithargsret{\bfcode{system\_user\_home}}{}{}
Returns home directory of the user.

\end{fulllineitems}

\index{system\_user\_id() (limeade.system.models.Person method)}

\begin{fulllineitems}
\phantomsection\label{api/system:limeade.system.models.Person.system_user_id}\pysiglinewithargsret{\bfcode{system\_user\_id}}{}{}
Returns the user id.

\end{fulllineitems}

\index{system\_user\_name() (limeade.system.models.Person method)}

\begin{fulllineitems}
\phantomsection\label{api/system:limeade.system.models.Person.system_user_name}\pysiglinewithargsret{\bfcode{system\_user\_name}}{}{}
Returns unicode representation of the username.

\end{fulllineitems}

\index{username() (limeade.system.models.Person method)}

\begin{fulllineitems}
\phantomsection\label{api/system:limeade.system.models.Person.username}\pysiglinewithargsret{\bfcode{username}}{}{}
Returns the username of the user.

\end{fulllineitems}


\end{fulllineitems}

\index{Product (class in limeade.system.models)}

\begin{fulllineitems}
\phantomsection\label{api/system:limeade.system.models.Product}\pysiglinewithargsret{\strong{class }\code{limeade.system.models.}\bfcode{Product}}{\emph{*args}, \emph{**kwargs}}{}
Defines a product which belongs to a person.
\begin{quote}\begin{description}
\item[{Parameters}] \leavevmode\begin{itemize}
\item {} 
\textbf{name} -- name of the product

\item {} 
\textbf{personalized} -- true if its, othwerwise false

\item {} 
\textbf{owner} -- the person who owns this product

\end{itemize}

\item[{Example }] \leavevmode
\begin{Verbatim}[commandchars=\\\{\}]
\PYG{g+gp}{\PYGZgt{}\PYGZgt{}\PYGZgt{} }\PYG{k+kn}{from} \PYG{n+nn}{limeade.system.models} \PYG{k+kn}{import} \PYG{n}{Product}
\PYG{g+gp}{\PYGZgt{}\PYGZgt{}\PYGZgt{} }\PYG{n}{product} \PYG{o}{=} \PYG{n}{Product}\PYG{o}{.}\PYG{n}{objects}\PYG{o}{.}\PYG{n}{get}\PYG{p}{(}\PYG{n}{pk}\PYG{o}{=}\PYG{l+m+mi}{1}\PYG{p}{)}
\PYG{g+gp}{\PYGZgt{}\PYGZgt{}\PYGZgt{} }\PYG{n}{product}
\PYG{g+go}{\PYGZlt{}Product: Testproduct\PYGZgt{}}
\end{Verbatim}

\end{description}\end{quote}
\index{has\_add\_permission() (limeade.system.models.Product method)}

\begin{fulllineitems}
\phantomsection\label{api/system:limeade.system.models.Product.has_add_permission}\pysiglinewithargsret{\bfcode{has\_add\_permission}}{\emph{request}}{}
Returns true if user has permission to add.

\end{fulllineitems}

\index{has\_change\_permission() (limeade.system.models.Product method)}

\begin{fulllineitems}
\phantomsection\label{api/system:limeade.system.models.Product.has_change_permission}\pysiglinewithargsret{\bfcode{has\_change\_permission}}{\emph{request}, \emph{obj}}{}
Returns true if user has permission to change.

\end{fulllineitems}

\index{has\_delete\_permission() (limeade.system.models.Product method)}

\begin{fulllineitems}
\phantomsection\label{api/system:limeade.system.models.Product.has_delete_permission}\pysiglinewithargsret{\bfcode{has\_delete\_permission}}{\emph{request}, \emph{obj}}{}
Returns true if user has permission to delete.

\end{fulllineitems}

\index{queryset() (limeade.system.models.Product method)}

\begin{fulllineitems}
\phantomsection\label{api/system:limeade.system.models.Product.queryset}\pysiglinewithargsret{\bfcode{queryset}}{\emph{request}}{}
Returns a filtered queryset.

\end{fulllineitems}


\end{fulllineitems}

\index{create\_user\_profile() (in module limeade.system.models)}

\begin{fulllineitems}
\phantomsection\label{api/system:limeade.system.models.create_user_profile}\pysiglinewithargsret{\code{limeade.system.models.}\bfcode{create\_user\_profile}}{\emph{sender}, \emph{instance}, \emph{created}, \emph{**kwargs}}{}
This function creates a user profile for Django's get\_profile() method, 
after the user is registered. This works with signals.

\end{fulllineitems}



\subsection{Views}
\label{api/system:module-limeade.system.views}\label{api/system:views}\index{limeade.system.views (module)}
Views for limeade system
\index{account() (in module limeade.system.views)}

\begin{fulllineitems}
\phantomsection\label{api/system:limeade.system.views.account}\pysiglinewithargsret{\code{limeade.system.views.}\bfcode{account}}{\emph{request}, \emph{*args}, \emph{**kwargs}}{}
Display basic account information.
\begin{quote}\begin{description}
\item[{Parameters}] \leavevmode
\textbf{request} -- the request object

\item[{Returns}] \leavevmode
the account template

\end{description}\end{quote}

\end{fulllineitems}

\index{contract\_add() (in module limeade.system.views)}

\begin{fulllineitems}
\phantomsection\label{api/system:limeade.system.views.contract_add}\pysiglinewithargsret{\code{limeade.system.views.}\bfcode{contract\_add}}{\emph{request}, \emph{*args}, \emph{**kwargs}}{}
Add a new contract for a customer.
\begin{quote}\begin{description}
\item[{Parameters}] \leavevmode\begin{itemize}
\item {} 
\textbf{request} -- the request object

\item {} 
\textbf{slug} -- the id of the customer

\end{itemize}

\item[{Returns}] \leavevmode
template for adding a contract

\end{description}\end{quote}

\end{fulllineitems}

\index{contract\_customize() (in module limeade.system.views)}

\begin{fulllineitems}
\phantomsection\label{api/system:limeade.system.views.contract_customize}\pysiglinewithargsret{\code{limeade.system.views.}\bfcode{contract\_customize}}{\emph{request}, \emph{*args}, \emph{**kwargs}}{}
Personalize a plan for a customer.
\begin{quote}\begin{description}
\item[{Parameters}] \leavevmode\begin{itemize}
\item {} 
\textbf{request} -- the request object

\item {} 
\textbf{slug} -- the id of the customer

\item {} 
\textbf{contract\_id} -- the id of the contract

\end{itemize}

\item[{Returns}] \leavevmode
edit form template

\end{description}\end{quote}

\end{fulllineitems}

\index{contract\_delete() (in module limeade.system.views)}

\begin{fulllineitems}
\phantomsection\label{api/system:limeade.system.views.contract_delete}\pysiglinewithargsret{\code{limeade.system.views.}\bfcode{contract\_delete}}{\emph{request}, \emph{*args}, \emph{**kwargs}}{}
Remove a contract.
\begin{quote}\begin{description}
\item[{Parameters}] \leavevmode\begin{itemize}
\item {} 
\textbf{request} -- the request object

\item {} 
\textbf{slug} -- the id of the customer

\item {} 
\textbf{contract\_id} -- the id of the contract

\end{itemize}

\item[{Returns}] \leavevmode
redirect to customer view

\end{description}\end{quote}

\end{fulllineitems}

\index{customer\_add() (in module limeade.system.views)}

\begin{fulllineitems}
\phantomsection\label{api/system:limeade.system.views.customer_add}\pysiglinewithargsret{\code{limeade.system.views.}\bfcode{customer\_add}}{\emph{request}, \emph{*args}, \emph{**kwargs}}{}
Form to add a new customer.
\begin{quote}\begin{description}
\item[{Parameters}] \leavevmode
\textbf{request} -- the request object

\item[{Returns}] \leavevmode
the add a customer template

\end{description}\end{quote}

\end{fulllineitems}

\index{customer\_delete() (in module limeade.system.views)}

\begin{fulllineitems}
\phantomsection\label{api/system:limeade.system.views.customer_delete}\pysiglinewithargsret{\code{limeade.system.views.}\bfcode{customer\_delete}}{\emph{request}, \emph{*args}, \emph{**kwargs}}{}
Remove a customer.
\begin{quote}\begin{description}
\item[{Parameters}] \leavevmode\begin{itemize}
\item {} 
\textbf{request} -- the request object

\item {} 
\textbf{slug} -- the id of the customer

\end{itemize}

\item[{Returns}] \leavevmode
redirect to customer list

\end{description}\end{quote}

\end{fulllineitems}

\index{customer\_edit() (in module limeade.system.views)}

\begin{fulllineitems}
\phantomsection\label{api/system:limeade.system.views.customer_edit}\pysiglinewithargsret{\code{limeade.system.views.}\bfcode{customer\_edit}}{\emph{request}, \emph{*args}, \emph{**kwargs}}{}
Edit details of a customer.
\begin{quote}\begin{description}
\item[{Parameters}] \leavevmode\begin{itemize}
\item {} 
\textbf{request} -- the request object

\item {} 
\textbf{slug} -- the id of the customer

\end{itemize}

\item[{Returns}] \leavevmode
a edit form template

\end{description}\end{quote}

\end{fulllineitems}

\index{customer\_list() (in module limeade.system.views)}

\begin{fulllineitems}
\phantomsection\label{api/system:limeade.system.views.customer_list}\pysiglinewithargsret{\code{limeade.system.views.}\bfcode{customer\_list}}{\emph{request}, \emph{*args}, \emph{**kwargs}}{}
Show a list of customers.
\begin{quote}\begin{description}
\item[{Parameters}] \leavevmode
\textbf{request} -- the request object

\item[{Returns}] \leavevmode
the customer template

\end{description}\end{quote}

\end{fulllineitems}

\index{customer\_manage() (in module limeade.system.views)}

\begin{fulllineitems}
\phantomsection\label{api/system:limeade.system.views.customer_manage}\pysiglinewithargsret{\code{limeade.system.views.}\bfcode{customer\_manage}}{\emph{request}, \emph{*args}, \emph{**kwargs}}{}
Switch the current user to the customer. This allowes one to interact with 
the interface as if one was logged into the customers account.
\begin{quote}\begin{description}
\item[{Parameters}] \leavevmode\begin{itemize}
\item {} 
\textbf{request} -- the request object

\item {} 
\textbf{slug} -- the id of the customer

\end{itemize}

\item[{Returns}] \leavevmode
redirect to ressources

\end{description}\end{quote}

\end{fulllineitems}

\index{customer\_manage\_return() (in module limeade.system.views)}

\begin{fulllineitems}
\phantomsection\label{api/system:limeade.system.views.customer_manage_return}\pysiglinewithargsret{\code{limeade.system.views.}\bfcode{customer\_manage\_return}}{\emph{request}, \emph{*args}, \emph{**kwargs}}{}
Return to the original user.
\begin{quote}\begin{description}
\item[{Parameters}] \leavevmode
\textbf{request} -- the request object

\item[{Returns}] \leavevmode
redirect to ressources

\end{description}\end{quote}

\end{fulllineitems}

\index{customer\_view() (in module limeade.system.views)}

\begin{fulllineitems}
\phantomsection\label{api/system:limeade.system.views.customer_view}\pysiglinewithargsret{\code{limeade.system.views.}\bfcode{customer\_view}}{\emph{request}, \emph{*args}, \emph{**kwargs}}{}
View details of a customer.
\begin{quote}\begin{description}
\item[{Parameters}] \leavevmode\begin{itemize}
\item {} 
\textbf{request} -- the request object

\item {} 
\textbf{slug} -- the id of the customer

\end{itemize}

\item[{Returns}] \leavevmode
the customer detail template

\end{description}\end{quote}

\end{fulllineitems}

\index{domain\_add() (in module limeade.system.views)}

\begin{fulllineitems}
\phantomsection\label{api/system:limeade.system.views.domain_add}\pysiglinewithargsret{\code{limeade.system.views.}\bfcode{domain\_add}}{\emph{request}, \emph{slug}, \emph{contract\_id}}{}
Add a new domain.
\begin{quote}\begin{description}
\item[{Parameters}] \leavevmode\begin{itemize}
\item {} 
\textbf{request} -- the request object

\item {} 
\textbf{slug} -- the id of the customer

\item {} 
\textbf{contract\_id} -- the id of the contract

\end{itemize}

\item[{Returns}] \leavevmode
a edit form template

\end{description}\end{quote}

\end{fulllineitems}

\index{domain\_delete() (in module limeade.system.views)}

\begin{fulllineitems}
\phantomsection\label{api/system:limeade.system.views.domain_delete}\pysiglinewithargsret{\code{limeade.system.views.}\bfcode{domain\_delete}}{\emph{request}, \emph{slug}, \emph{contract\_id}, \emph{domain\_id}}{}
Remove a domain.
\begin{quote}\begin{description}
\item[{Parameters}] \leavevmode\begin{itemize}
\item {} 
\textbf{request} -- the request object

\item {} 
\textbf{slug} -- the id of the customer

\item {} 
\textbf{contract\_id} -- the id of the contract

\item {} 
\textbf{domain\_id} -- the id of the domain

\end{itemize}

\item[{Returns}] \leavevmode
redirect to customer view

\end{description}\end{quote}

\end{fulllineitems}

\index{product\_add() (in module limeade.system.views)}

\begin{fulllineitems}
\phantomsection\label{api/system:limeade.system.views.product_add}\pysiglinewithargsret{\code{limeade.system.views.}\bfcode{product\_add}}{\emph{request}, \emph{*args}, \emph{**kwargs}}{}
Create a new product. A product is a combination of different limitsets.
\begin{quote}\begin{description}
\item[{Parameters}] \leavevmode
\textbf{request} -- the request object

\item[{Returns}] \leavevmode
a edit form template

\end{description}\end{quote}

\end{fulllineitems}

\index{product\_delete() (in module limeade.system.views)}

\begin{fulllineitems}
\phantomsection\label{api/system:limeade.system.views.product_delete}\pysiglinewithargsret{\code{limeade.system.views.}\bfcode{product\_delete}}{\emph{request}, \emph{*args}, \emph{**kwargs}}{}
Remove a product.
\begin{quote}\begin{description}
\item[{Parameters}] \leavevmode\begin{itemize}
\item {} 
\textbf{request} -- the request object

\item {} 
\textbf{slug} -- the id of the product

\end{itemize}

\item[{Returns}] \leavevmode
a edit form template

\end{description}\end{quote}

\end{fulllineitems}

\index{product\_edit() (in module limeade.system.views)}

\begin{fulllineitems}
\phantomsection\label{api/system:limeade.system.views.product_edit}\pysiglinewithargsret{\code{limeade.system.views.}\bfcode{product\_edit}}{\emph{request}, \emph{*args}, \emph{**kwargs}}{}
Edit a product.
\begin{quote}\begin{description}
\item[{Parameters}] \leavevmode\begin{itemize}
\item {} 
\textbf{request} -- the request object

\item {} 
\textbf{slug} -- the id of the product

\item {} 
\textbf{next} -- redirect to view

\end{itemize}

\item[{Returns}] \leavevmode
a edit form template

\end{description}\end{quote}

\end{fulllineitems}

\index{product\_list() (in module limeade.system.views)}

\begin{fulllineitems}
\phantomsection\label{api/system:limeade.system.views.product_list}\pysiglinewithargsret{\code{limeade.system.views.}\bfcode{product\_list}}{\emph{request}, \emph{*args}, \emph{**kwargs}}{}
Show a list of products.
\begin{quote}\begin{description}
\item[{Parameters}] \leavevmode
\textbf{request} -- the request object

\item[{Returns}] \leavevmode
a list of products

\end{description}\end{quote}

\end{fulllineitems}

\index{ressources() (in module limeade.system.views)}

\begin{fulllineitems}
\phantomsection\label{api/system:limeade.system.views.ressources}\pysiglinewithargsret{\code{limeade.system.views.}\bfcode{ressources}}{\emph{request}, \emph{*args}, \emph{**kwargs}}{}
Display the aggregated ressources available to the current user. This 
depends on the products a user has subscribed to.
\begin{quote}\begin{description}
\item[{Parameters}] \leavevmode
\textbf{request} -- the request object

\item[{Returns}] \leavevmode
the ressource template

\end{description}\end{quote}

\end{fulllineitems}



\section{Web}
\label{api/web:web}\label{api/web::doc}

\subsection{Models}
\label{api/web:models}\label{api/web:module-limeade.web.models}\index{limeade.web.models (module)}
Models for limeade web
\index{DefaultVHost (class in limeade.web.models)}

\begin{fulllineitems}
\phantomsection\label{api/web:limeade.web.models.DefaultVHost}\pysiglinewithargsret{\strong{class }\code{limeade.web.models.}\bfcode{DefaultVHost}}{\emph{*args}, \emph{**kwargs}}{}
Saves the default VHost.
\begin{quote}\begin{description}
\item[{Parameters}] \leavevmode\begin{itemize}
\item {} 
\textbf{doamin} -- the foreign key to the domain

\item {} 
\textbf{vhost} -- the foreign key to the vhost

\end{itemize}

\end{description}\end{quote}

\end{fulllineitems}

\index{HTTPRedirect (class in limeade.web.models)}

\begin{fulllineitems}
\phantomsection\label{api/web:limeade.web.models.HTTPRedirect}\pysiglinewithargsret{\strong{class }\code{limeade.web.models.}\bfcode{HTTPRedirect}}{\emph{*args}, \emph{**kwargs}}{}
Creates http redirects
\begin{quote}\begin{description}
\item[{Parameters}] \leavevmode\begin{itemize}
\item {} 
\textbf{name} -- the name of the redirect

\item {} 
\textbf{doamin} -- the associated domain

\item {} 
\textbf{to} -- the redirect to

\end{itemize}

\end{description}\end{quote}

\end{fulllineitems}

\index{Limitset (class in limeade.web.models)}

\begin{fulllineitems}
\phantomsection\label{api/web:limeade.web.models.Limitset}\pysiglinewithargsret{\strong{class }\code{limeade.web.models.}\bfcode{Limitset}}{\emph{*args}, \emph{**kwargs}}{}
Maximum limit available.
\begin{quote}\begin{description}
\item[{Parameters}] \leavevmode\begin{itemize}
\item {} 
\textbf{products} -- the foreign key to the product

\item {} 
\textbf{vhosts} -- maximum vhosts

\item {} 
\textbf{redirects} -- maximum redirects

\item {} 
\textbf{webspace} -- maximum webspace

\item {} 
\textbf{cputime} -- maximum cputime

\end{itemize}

\end{description}\end{quote}
\index{utilization() (limeade.web.models.Limitset static method)}

\begin{fulllineitems}
\phantomsection\label{api/web:limeade.web.models.Limitset.utilization}\pysiglinewithargsret{\strong{static }\bfcode{utilization}}{\emph{user}, \emph{ressource}}{}
Filters specific ressources.

\end{fulllineitems}


\end{fulllineitems}

\index{PoolIP (class in limeade.web.models)}

\begin{fulllineitems}
\phantomsection\label{api/web:limeade.web.models.PoolIP}\pysiglinewithargsret{\strong{class }\code{limeade.web.models.}\bfcode{PoolIP}}{\emph{*args}, \emph{**kwargs}}{}
The IP address pool.
\begin{quote}\begin{description}
\item[{Parameters}] \leavevmode\begin{itemize}
\item {} 
\textbf{ip} -- the IP

\item {} 
\textbf{region} -- the foreign key to the region

\end{itemize}

\item[{Example }] \leavevmode
\begin{Verbatim}[commandchars=\\\{\}]
\PYG{g+gp}{\PYGZgt{}\PYGZgt{}\PYGZgt{} }\PYG{k+kn}{from} \PYG{n+nn}{limeade.web.models} \PYG{k+kn}{import} \PYG{n}{PoolIP}
\PYG{g+gp}{\PYGZgt{}\PYGZgt{}\PYGZgt{} }\PYG{n}{ip} \PYG{o}{=} \PYG{n}{PoolIP}\PYG{o}{.}\PYG{n}{objects}\PYG{o}{.}\PYG{n}{get}\PYG{p}{(}\PYG{n}{pk}\PYG{o}{=}\PYG{l+m+mi}{1}\PYG{p}{)}
\PYG{g+gp}{\PYGZgt{}\PYGZgt{}\PYGZgt{} }\PYG{n}{ip}
\PYG{g+go}{\PYGZlt{}Person: 127.0.0.1\PYGZgt{}}
\end{Verbatim}

\end{description}\end{quote}

\end{fulllineitems}

\index{SSLCert (class in limeade.web.models)}

\begin{fulllineitems}
\phantomsection\label{api/web:limeade.web.models.SSLCert}\pysiglinewithargsret{\strong{class }\code{limeade.web.models.}\bfcode{SSLCert}}{\emph{*args}, \emph{**kwargs}}{}
Creates SSL Certifications
\begin{quote}\begin{description}
\item[{Parameters}] \leavevmode\begin{itemize}
\item {} 
\textbf{owner} -- the user

\item {} 
\textbf{serial} -- the ssl cert

\item {} 
\textbf{valid\_not\_before} -- date of valid

\item {} 
\textbf{valid\_not\_after} -- date of valid

\item {} 
\textbf{subject} -- topic of cert

\item {} 
\textbf{cn} -- cert number

\item {} 
\textbf{issuer} -- the person

\item {} 
\textbf{cert} -- the cert

\item {} 
\textbf{key} -- key of the cert

\item {} 
\textbf{ca} -- the ca of the cert

\item {} 
\textbf{ip} -- the foreign key to the ip

\end{itemize}

\end{description}\end{quote}
\index{set\_cert() (limeade.web.models.SSLCert method)}

\begin{fulllineitems}
\phantomsection\label{api/web:limeade.web.models.SSLCert.set_cert}\pysiglinewithargsret{\bfcode{set\_cert}}{\emph{cert}, \emph{key}, \emph{ca}}{}
Saves the certification.

\end{fulllineitems}


\end{fulllineitems}

\index{VHost (class in limeade.web.models)}

\begin{fulllineitems}
\phantomsection\label{api/web:limeade.web.models.VHost}\pysiglinewithargsret{\strong{class }\code{limeade.web.models.}\bfcode{VHost}}{\emph{*args}, \emph{**kwargs}}{}
Creates a vhost.
\begin{quote}\begin{description}
\item[{Parameters}] \leavevmode\begin{itemize}
\item {} 
\textbf{name} -- the name of the Vhost

\item {} 
\textbf{domain} -- foreign key to the domain

\item {} 
\textbf{style} -- php or python vhost

\item {} 
\textbf{cert} -- the ssl cert

\end{itemize}

\item[{Example }] \leavevmode
\begin{Verbatim}[commandchars=\\\{\}]
\PYG{g+gp}{\PYGZgt{}\PYGZgt{}\PYGZgt{} }\PYG{k+kn}{from} \PYG{n+nn}{limeade.web.models} \PYG{k+kn}{import} \PYG{n}{VHost}
\PYG{g+gp}{\PYGZgt{}\PYGZgt{}\PYGZgt{} }\PYG{n}{vhost} \PYG{o}{=} \PYG{n}{VHost}\PYG{o}{.}\PYG{n}{objects}\PYG{o}{.}\PYG{n}{get}\PYG{p}{(}\PYG{n}{pk}\PYG{o}{=}\PYG{l+m+mi}{1}\PYG{p}{)}
\PYG{g+gp}{\PYGZgt{}\PYGZgt{}\PYGZgt{} }\PYG{n}{vhost}
\PYG{g+go}{\PYGZlt{}Person: Test VHost.testdomain.de\PYGZgt{}}
\end{Verbatim}

\end{description}\end{quote}

\end{fulllineitems}

\index{get\_vhosts() (in module limeade.web.models)}

\begin{fulllineitems}
\phantomsection\label{api/web:limeade.web.models.get_vhosts}\pysiglinewithargsret{\code{limeade.web.models.}\bfcode{get\_vhosts}}{\emph{user}}{}
Returns all vhosts that belongs to one user.

\end{fulllineitems}



\subsection{Views}
\label{api/web:module-limeade.web.views}\label{api/web:views}\index{limeade.web.views (module)}
Views for limeade web
\index{poolip\_add() (in module limeade.web.views)}

\begin{fulllineitems}
\phantomsection\label{api/web:limeade.web.views.poolip_add}\pysiglinewithargsret{\code{limeade.web.views.}\bfcode{poolip\_add}}{\emph{request}, \emph{*args}, \emph{**kwargs}}{}
Adds an IP address.
\begin{quote}\begin{description}
\item[{Parameters}] \leavevmode
\textbf{request} -- the request object

\item[{Returns}] \leavevmode
an edit form template

\end{description}\end{quote}

\end{fulllineitems}

\index{poolip\_list() (in module limeade.web.views)}

\begin{fulllineitems}
\phantomsection\label{api/web:limeade.web.views.poolip_list}\pysiglinewithargsret{\code{limeade.web.views.}\bfcode{poolip\_list}}{\emph{request}, \emph{*args}, \emph{**kwargs}}{}
Lists IP addresses.
\begin{quote}\begin{description}
\item[{Parameters}] \leavevmode
\textbf{request} -- the request object

\item[{Returns}] \leavevmode
ip address pool template

\end{description}\end{quote}

\end{fulllineitems}

\index{redirect\_add() (in module limeade.web.views)}

\begin{fulllineitems}
\phantomsection\label{api/web:limeade.web.views.redirect_add}\pysiglinewithargsret{\code{limeade.web.views.}\bfcode{redirect\_add}}{\emph{request}, \emph{*args}, \emph{**kwargs}}{}
Create a new HTTP redirect.
\begin{quote}\begin{description}
\item[{Parameters}] \leavevmode
\textbf{request} -- the request object

\item[{Returns}] \leavevmode
an edit form template

\end{description}\end{quote}

\end{fulllineitems}

\index{redirect\_delete() (in module limeade.web.views)}

\begin{fulllineitems}
\phantomsection\label{api/web:limeade.web.views.redirect_delete}\pysiglinewithargsret{\code{limeade.web.views.}\bfcode{redirect\_delete}}{\emph{request}, \emph{*args}, \emph{**kwargs}}{}
Remove a HTTP Redirect.
\begin{quote}\begin{description}
\item[{Parameters}] \leavevmode\begin{itemize}
\item {} 
\textbf{request} -- the request object

\item {} 
\textbf{slug} -- the id of the redirect

\end{itemize}

\item[{Returns}] \leavevmode
redirect to http redirects list

\end{description}\end{quote}

\end{fulllineitems}

\index{redirect\_edit() (in module limeade.web.views)}

\begin{fulllineitems}
\phantomsection\label{api/web:limeade.web.views.redirect_edit}\pysiglinewithargsret{\code{limeade.web.views.}\bfcode{redirect\_edit}}{\emph{request}, \emph{*args}, \emph{**kwargs}}{}
Edit a HTTP redirect.
\begin{quote}\begin{description}
\item[{Parameters}] \leavevmode\begin{itemize}
\item {} 
\textbf{request} -- the request object

\item {} 
\textbf{slug} -- the id of the redirect

\end{itemize}

\item[{Returns}] \leavevmode
an edit form template

\end{description}\end{quote}

\end{fulllineitems}

\index{redirect\_list() (in module limeade.web.views)}

\begin{fulllineitems}
\phantomsection\label{api/web:limeade.web.views.redirect_list}\pysiglinewithargsret{\code{limeade.web.views.}\bfcode{redirect\_list}}{\emph{request}, \emph{*args}, \emph{**kwargs}}{}
List HTTP redirects.
\begin{quote}\begin{description}
\item[{Parameters}] \leavevmode
\textbf{request} -- the request object

\item[{Returns}] \leavevmode
list of http redirects

\end{description}\end{quote}

\end{fulllineitems}

\index{sslcert\_add() (in module limeade.web.views)}

\begin{fulllineitems}
\phantomsection\label{api/web:limeade.web.views.sslcert_add}\pysiglinewithargsret{\code{limeade.web.views.}\bfcode{sslcert\_add}}{\emph{request}, \emph{*args}, \emph{**kwargs}}{}
Add a new SSL certificate.
\begin{quote}\begin{description}
\item[{Parameters}] \leavevmode
\textbf{request} -- the request object

\item[{Returns}] \leavevmode
an edit form template

\end{description}\end{quote}

\end{fulllineitems}

\index{sslcert\_delete() (in module limeade.web.views)}

\begin{fulllineitems}
\phantomsection\label{api/web:limeade.web.views.sslcert_delete}\pysiglinewithargsret{\code{limeade.web.views.}\bfcode{sslcert\_delete}}{\emph{request}, \emph{*args}, \emph{**kwargs}}{}
Remove a SSL certificate.
\begin{quote}\begin{description}
\item[{Parameters}] \leavevmode\begin{itemize}
\item {} 
\textbf{request} -- the request object

\item {} 
\textbf{slug} -- the id of the ssl cert

\end{itemize}

\item[{Returns}] \leavevmode
redirects to ssl cert list

\end{description}\end{quote}

\end{fulllineitems}

\index{sslcert\_list() (in module limeade.web.views)}

\begin{fulllineitems}
\phantomsection\label{api/web:limeade.web.views.sslcert_list}\pysiglinewithargsret{\code{limeade.web.views.}\bfcode{sslcert\_list}}{\emph{request}, \emph{*args}, \emph{**kwargs}}{}
List a users SSL certificates.
\begin{quote}\begin{description}
\item[{Parameters}] \leavevmode
\textbf{request} -- the request object

\item[{Returns}] \leavevmode
a list of ssl certificates

\end{description}\end{quote}

\end{fulllineitems}

\index{vhost\_add() (in module limeade.web.views)}

\begin{fulllineitems}
\phantomsection\label{api/web:limeade.web.views.vhost_add}\pysiglinewithargsret{\code{limeade.web.views.}\bfcode{vhost\_add}}{\emph{request}, \emph{*args}, \emph{**kwargs}}{}
Create a new web-vhost.
\begin{quote}\begin{description}
\item[{Parameters}] \leavevmode
\textbf{request} -- the request object

\item[{Returns}] \leavevmode
an edit form template

\end{description}\end{quote}

\end{fulllineitems}

\index{vhost\_catchall\_delete() (in module limeade.web.views)}

\begin{fulllineitems}
\phantomsection\label{api/web:limeade.web.views.vhost_catchall_delete}\pysiglinewithargsret{\code{limeade.web.views.}\bfcode{vhost\_catchall\_delete}}{\emph{request}, \emph{*args}, \emph{**kwargs}}{}
Disable catch-all for this domain.
\begin{quote}\begin{description}
\item[{Parameters}] \leavevmode\begin{itemize}
\item {} 
\textbf{request} -- the request object

\item {} 
\textbf{slug} -- the id of the vhost

\end{itemize}

\item[{Returns}] \leavevmode
redirects to vhost list

\end{description}\end{quote}

\end{fulllineitems}

\index{vhost\_catchall\_set() (in module limeade.web.views)}

\begin{fulllineitems}
\phantomsection\label{api/web:limeade.web.views.vhost_catchall_set}\pysiglinewithargsret{\code{limeade.web.views.}\bfcode{vhost\_catchall\_set}}{\emph{request}, \emph{*args}, \emph{**kwargs}}{}
Set this vhost as catch-all for the domain.
\begin{quote}\begin{description}
\item[{Parameters}] \leavevmode\begin{itemize}
\item {} 
\textbf{request} -- the request object

\item {} 
\textbf{slug} -- the id of the vhost

\end{itemize}

\item[{Returns}] \leavevmode
redirects to vhost list

\end{description}\end{quote}

\end{fulllineitems}

\index{vhost\_delete() (in module limeade.web.views)}

\begin{fulllineitems}
\phantomsection\label{api/web:limeade.web.views.vhost_delete}\pysiglinewithargsret{\code{limeade.web.views.}\bfcode{vhost\_delete}}{\emph{request}, \emph{*args}, \emph{**kwargs}}{}
Remove a web-vhost.
\begin{quote}\begin{description}
\item[{Parameters}] \leavevmode\begin{itemize}
\item {} 
\textbf{request} -- the request object

\item {} 
\textbf{slug} -- the id of the vhost

\end{itemize}

\item[{Returns}] \leavevmode
redirects to vhost list

\end{description}\end{quote}

\end{fulllineitems}

\index{vhost\_edit() (in module limeade.web.views)}

\begin{fulllineitems}
\phantomsection\label{api/web:limeade.web.views.vhost_edit}\pysiglinewithargsret{\code{limeade.web.views.}\bfcode{vhost\_edit}}{\emph{request}, \emph{*args}, \emph{**kwargs}}{}
Edit details of a web-vhost.
\begin{quote}\begin{description}
\item[{Parameters}] \leavevmode\begin{itemize}
\item {} 
\textbf{request} -- the request object

\item {} 
\textbf{slug} -- the id of the vhost

\end{itemize}

\item[{Returns}] \leavevmode
an edit form template

\end{description}\end{quote}

\end{fulllineitems}

\index{vhost\_list() (in module limeade.web.views)}

\begin{fulllineitems}
\phantomsection\label{api/web:limeade.web.views.vhost_list}\pysiglinewithargsret{\code{limeade.web.views.}\bfcode{vhost\_list}}{\emph{request}, \emph{*args}, \emph{**kwargs}}{}
List of web-vhosts.
\begin{quote}\begin{description}
\item[{Parameters}] \leavevmode
\textbf{request} -- the request object

\item[{Returns}] \leavevmode
list of web vhosts

\end{description}\end{quote}

\end{fulllineitems}



\section{MySQL}
\label{api/mysql::doc}\label{api/mysql:mysql}

\subsection{Models}
\label{api/mysql:models}\label{api/mysql:module-limeade.mysql.models}\index{limeade.mysql.models (module)}
Models for limeade mysql
\index{Limitset (class in limeade.mysql.models)}

\begin{fulllineitems}
\phantomsection\label{api/mysql:limeade.mysql.models.Limitset}\pysiglinewithargsret{\strong{class }\code{limeade.mysql.models.}\bfcode{Limitset}}{\emph{*args}, \emph{**kwargs}}{}
Saves maximum available databases.
\begin{quote}\begin{description}
\item[{Parameters}] \leavevmode\begin{itemize}
\item {} 
\textbf{product} -- the foreign key to the product

\item {} 
\textbf{dbs} -- the maxmimum databases

\end{itemize}

\end{description}\end{quote}

\end{fulllineitems}



\subsection{Views}
\label{api/mysql:module-limeade.mysql.views}\label{api/mysql:views}\index{limeade.mysql.views (module)}
Views for limeade mysql
\index{db\_add() (in module limeade.mysql.views)}

\begin{fulllineitems}
\phantomsection\label{api/mysql:limeade.mysql.views.db_add}\pysiglinewithargsret{\code{limeade.mysql.views.}\bfcode{db\_add}}{\emph{request}, \emph{*args}, \emph{**kwargs}}{}
Form to add a new Database. Each database has a respective user with its 
own credentials.
\begin{quote}\begin{description}
\item[{Parameters}] \leavevmode
\textbf{request} -- the request object

\item[{Returns}] \leavevmode
an edit form template

\end{description}\end{quote}

\end{fulllineitems}

\index{db\_delete() (in module limeade.mysql.views)}

\begin{fulllineitems}
\phantomsection\label{api/mysql:limeade.mysql.views.db_delete}\pysiglinewithargsret{\code{limeade.mysql.views.}\bfcode{db\_delete}}{\emph{request}, \emph{*args}, \emph{**kwargs}}{}
Drop a database.
\begin{quote}\begin{description}
\item[{Parameters}] \leavevmode\begin{itemize}
\item {} 
\textbf{request} -- the request object

\item {} 
\textbf{slug} -- the id of the database

\end{itemize}

\item[{Returns}] \leavevmode
redirect to the database list

\end{description}\end{quote}

\end{fulllineitems}

\index{db\_edit() (in module limeade.mysql.views)}

\begin{fulllineitems}
\phantomsection\label{api/mysql:limeade.mysql.views.db_edit}\pysiglinewithargsret{\code{limeade.mysql.views.}\bfcode{db\_edit}}{\emph{request}, \emph{*args}, \emph{**kwargs}}{}
Form to set a new password for the database.
\begin{quote}\begin{description}
\item[{Parameters}] \leavevmode\begin{itemize}
\item {} 
\textbf{request} -- the request object

\item {} 
\textbf{slug} -- the id of the database

\end{itemize}

\item[{Returns}] \leavevmode
an edit form template

\end{description}\end{quote}

\end{fulllineitems}

\index{db\_list() (in module limeade.mysql.views)}

\begin{fulllineitems}
\phantomsection\label{api/mysql:limeade.mysql.views.db_list}\pysiglinewithargsret{\code{limeade.mysql.views.}\bfcode{db\_list}}{\emph{request}, \emph{*args}, \emph{**kwargs}}{}
Show a list of the users databases.
\begin{quote}\begin{description}
\item[{Parameters}] \leavevmode
\textbf{request} -- the request object

\item[{Returns}] \leavevmode
a list of databases

\item[{Raises }] \leavevmode
TimeoutError

\end{description}\end{quote}

\end{fulllineitems}



\section{Mail}
\label{api/mail:mail}\label{api/mail::doc}

\subsection{Models}
\label{api/mail:models}\label{api/mail:module-limeade.mail.models}\index{limeade.mail.models (module)}
Models for limeade mail
\index{Account (class in limeade.mail.models)}

\begin{fulllineitems}
\phantomsection\label{api/mail:limeade.mail.models.Account}\pysiglinewithargsret{\strong{class }\code{limeade.mail.models.}\bfcode{Account}}{\emph{*args}, \emph{**kwargs}}{}
Creates a new mail account
\begin{quote}\begin{description}
\item[{Parameters}] \leavevmode\begin{itemize}
\item {} 
\textbf{name} -- the name of the account

\item {} 
\textbf{domain} -- the foreign key to the domain

\item {} 
\textbf{password} -- the password of the account

\end{itemize}

\end{description}\end{quote}
\index{set\_password() (limeade.mail.models.Account method)}

\begin{fulllineitems}
\phantomsection\label{api/mail:limeade.mail.models.Account.set_password}\pysiglinewithargsret{\bfcode{set\_password}}{\emph{password}}{}
Salt and hashes the password.

\end{fulllineitems}


\end{fulllineitems}

\index{Limitset (class in limeade.mail.models)}

\begin{fulllineitems}
\phantomsection\label{api/mail:limeade.mail.models.Limitset}\pysiglinewithargsret{\strong{class }\code{limeade.mail.models.}\bfcode{Limitset}}{\emph{*args}, \emph{**kwargs}}{}
Creates a limitset.
\begin{quote}\begin{description}
\item[{Parameters}] \leavevmode\begin{itemize}
\item {} 
\textbf{product} -- the foreign key to the product

\item {} 
\textbf{accounts} -- the maximum number of mail accounts

\item {} 
\textbf{redirects} -- the maximum number of mail redirects

\item {} 
\textbf{mailspace} -- the maximum number of space

\end{itemize}

\end{description}\end{quote}
\index{utilization() (limeade.mail.models.Limitset static method)}

\begin{fulllineitems}
\phantomsection\label{api/mail:limeade.mail.models.Limitset.utilization}\pysiglinewithargsret{\strong{static }\bfcode{utilization}}{\emph{user}, \emph{ressource}}{}
Returns the correct ressource.

\end{fulllineitems}


\end{fulllineitems}

\index{Redirect (class in limeade.mail.models)}

\begin{fulllineitems}
\phantomsection\label{api/mail:limeade.mail.models.Redirect}\pysiglinewithargsret{\strong{class }\code{limeade.mail.models.}\bfcode{Redirect}}{\emph{*args}, \emph{**kwargs}}{}
Creates a new mail redirect
\begin{quote}\begin{description}
\item[{Parameters}] \leavevmode\begin{itemize}
\item {} 
\textbf{name} -- the name of the redirect

\item {} 
\textbf{domain} -- the foreign key to the domain

\item {} 
\textbf{to} -- redirect to

\end{itemize}

\end{description}\end{quote}

\end{fulllineitems}



\subsection{Views}
\label{api/mail:module-limeade.mail.views}\label{api/mail:views}\index{limeade.mail.views (module)}
Views for limeade mail
\index{account\_add() (in module limeade.mail.views)}

\begin{fulllineitems}
\phantomsection\label{api/mail:limeade.mail.views.account_add}\pysiglinewithargsret{\code{limeade.mail.views.}\bfcode{account\_add}}{\emph{request}, \emph{*args}, \emph{**kwargs}}{}
Add a new mail account.
\begin{quote}\begin{description}
\item[{Parameters}] \leavevmode
\textbf{request} -- the request object

\item[{Returns}] \leavevmode
an edit form template

\end{description}\end{quote}

\end{fulllineitems}

\index{account\_delete() (in module limeade.mail.views)}

\begin{fulllineitems}
\phantomsection\label{api/mail:limeade.mail.views.account_delete}\pysiglinewithargsret{\code{limeade.mail.views.}\bfcode{account\_delete}}{\emph{request}, \emph{*args}, \emph{**kwargs}}{}
Remove a mail account.
\begin{quote}\begin{description}
\item[{Parameters}] \leavevmode\begin{itemize}
\item {} 
\textbf{request} -- the request object

\item {} 
\textbf{slug} -- the id of the account

\end{itemize}

\item[{Returns}] \leavevmode
redirects to mail account list

\end{description}\end{quote}

\end{fulllineitems}

\index{account\_edit() (in module limeade.mail.views)}

\begin{fulllineitems}
\phantomsection\label{api/mail:limeade.mail.views.account_edit}\pysiglinewithargsret{\code{limeade.mail.views.}\bfcode{account\_edit}}{\emph{request}, \emph{*args}, \emph{**kwargs}}{}
Set a new password for an email account.
\begin{quote}\begin{description}
\item[{Parameters}] \leavevmode\begin{itemize}
\item {} 
\textbf{request} -- the request object

\item {} 
\textbf{slug} -- the id of the account

\end{itemize}

\item[{Returns}] \leavevmode
an edit form template

\end{description}\end{quote}

\end{fulllineitems}

\index{account\_list() (in module limeade.mail.views)}

\begin{fulllineitems}
\phantomsection\label{api/mail:limeade.mail.views.account_list}\pysiglinewithargsret{\code{limeade.mail.views.}\bfcode{account\_list}}{\emph{request}, \emph{*args}, \emph{**kwargs}}{}
List all mail accounts of a user.
\begin{quote}\begin{description}
\item[{Parameters}] \leavevmode
\textbf{request} -- the request object

\item[{Returns}] \leavevmode
a list of mail accounts

\end{description}\end{quote}

\end{fulllineitems}

\index{redirect\_add() (in module limeade.mail.views)}

\begin{fulllineitems}
\phantomsection\label{api/mail:limeade.mail.views.redirect_add}\pysiglinewithargsret{\code{limeade.mail.views.}\bfcode{redirect\_add}}{\emph{request}, \emph{*args}, \emph{**kwargs}}{}
Create a new mail redirect.
\begin{quote}\begin{description}
\item[{Parameters}] \leavevmode
\textbf{request} -- the request object

\item[{Returns}] \leavevmode
an edit form template

\end{description}\end{quote}

\end{fulllineitems}

\index{redirect\_delete() (in module limeade.mail.views)}

\begin{fulllineitems}
\phantomsection\label{api/mail:limeade.mail.views.redirect_delete}\pysiglinewithargsret{\code{limeade.mail.views.}\bfcode{redirect\_delete}}{\emph{request}, \emph{*args}, \emph{**kwargs}}{}
Remove a mail redirect.
\begin{quote}\begin{description}
\item[{Parameters}] \leavevmode\begin{itemize}
\item {} 
\textbf{request} -- the request object

\item {} 
\textbf{slug} -- the id of the redirect

\end{itemize}

\item[{Returns}] \leavevmode
redirects to a list of mail redirects

\end{description}\end{quote}

\end{fulllineitems}

\index{redirect\_edit() (in module limeade.mail.views)}

\begin{fulllineitems}
\phantomsection\label{api/mail:limeade.mail.views.redirect_edit}\pysiglinewithargsret{\code{limeade.mail.views.}\bfcode{redirect\_edit}}{\emph{request}, \emph{*args}, \emph{**kwargs}}{}
Edit a mail redirect.
\begin{quote}\begin{description}
\item[{Parameters}] \leavevmode\begin{itemize}
\item {} 
\textbf{request} -- the request object

\item {} 
\textbf{slug} -- the id of the redirect

\end{itemize}

\item[{Returns}] \leavevmode
an edit form template

\end{description}\end{quote}

\end{fulllineitems}

\index{redirect\_list() (in module limeade.mail.views)}

\begin{fulllineitems}
\phantomsection\label{api/mail:limeade.mail.views.redirect_list}\pysiglinewithargsret{\code{limeade.mail.views.}\bfcode{redirect\_list}}{\emph{request}, \emph{*args}, \emph{**kwargs}}{}
List all mail redirects.
\begin{quote}\begin{description}
\item[{Parameters}] \leavevmode
\textbf{request} -- the request object

\item[{Returns}] \leavevmode
a list of mail redirects

\end{description}\end{quote}

\end{fulllineitems}



\section{FTP}
\label{api/ftp:ftp}\label{api/ftp::doc}

\subsection{Models}
\label{api/ftp:models}\label{api/ftp:module-limeade.ftp.models}\index{limeade.ftp.models (module)}
Models for limeade ftp
\index{Account (class in limeade.ftp.models)}

\begin{fulllineitems}
\phantomsection\label{api/ftp:limeade.ftp.models.Account}\pysiglinewithargsret{\strong{class }\code{limeade.ftp.models.}\bfcode{Account}}{\emph{*args}, \emph{**kwargs}}{}
Creates an ftp account.
\begin{quote}\begin{description}
\item[{Parameters}] \leavevmode\begin{itemize}
\item {} 
\textbf{name} -- the name of the account

\item {} 
\textbf{password} -- the password

\item {} 
\textbf{vhost} -- the foreign key to the vhost

\end{itemize}

\end{description}\end{quote}
\index{save() (limeade.ftp.models.Account method)}

\begin{fulllineitems}
\phantomsection\label{api/ftp:limeade.ftp.models.Account.save}\pysiglinewithargsret{\bfcode{save}}{\emph{**kwargs}}{}
Saves the ftp account.

\end{fulllineitems}

\index{set\_password() (limeade.ftp.models.Account method)}

\begin{fulllineitems}
\phantomsection\label{api/ftp:limeade.ftp.models.Account.set_password}\pysiglinewithargsret{\bfcode{set\_password}}{\emph{password}}{}
Salt and hashes the password.

\end{fulllineitems}


\end{fulllineitems}

\index{Limitset (class in limeade.ftp.models)}

\begin{fulllineitems}
\phantomsection\label{api/ftp:limeade.ftp.models.Limitset}\pysiglinewithargsret{\strong{class }\code{limeade.ftp.models.}\bfcode{Limitset}}{\emph{*args}, \emph{**kwargs}}{}
Saves the maximum number of ftp accounts.
\begin{quote}\begin{description}
\item[{Parameters}] \leavevmode\begin{itemize}
\item {} 
\textbf{product} -- the foreign key to the product

\item {} 
\textbf{accounts} -- maximum number of ftp accounts

\end{itemize}

\end{description}\end{quote}
\index{utilization() (limeade.ftp.models.Limitset static method)}

\begin{fulllineitems}
\phantomsection\label{api/ftp:limeade.ftp.models.Limitset.utilization}\pysiglinewithargsret{\strong{static }\bfcode{utilization}}{\emph{user}, \emph{ressource}}{}
Returns the correct ressource.

\end{fulllineitems}


\end{fulllineitems}



\subsection{Views}
\label{api/ftp:module-limeade.ftp.views}\label{api/ftp:views}\index{limeade.ftp.views (module)}
Views for limeade ftp
\index{account\_add() (in module limeade.ftp.views)}

\begin{fulllineitems}
\phantomsection\label{api/ftp:limeade.ftp.views.account_add}\pysiglinewithargsret{\code{limeade.ftp.views.}\bfcode{account\_add}}{\emph{request}, \emph{*args}, \emph{**kwargs}}{}
Add a new FTP account.
\begin{quote}\begin{description}
\item[{Parameters}] \leavevmode
\textbf{request} -- the request object

\item[{Returns}] \leavevmode
an edit form template

\end{description}\end{quote}

\end{fulllineitems}

\index{account\_delete() (in module limeade.ftp.views)}

\begin{fulllineitems}
\phantomsection\label{api/ftp:limeade.ftp.views.account_delete}\pysiglinewithargsret{\code{limeade.ftp.views.}\bfcode{account\_delete}}{\emph{request}, \emph{*args}, \emph{**kwargs}}{}
Remove an ftp account.
\begin{quote}\begin{description}
\item[{Parameters}] \leavevmode\begin{itemize}
\item {} 
\textbf{request} -- the request object

\item {} 
\textbf{slug} -- the id of the ftp account

\end{itemize}

\item[{Returns}] \leavevmode
redirects to a list of ftp accounts

\end{description}\end{quote}

\end{fulllineitems}

\index{account\_edit() (in module limeade.ftp.views)}

\begin{fulllineitems}
\phantomsection\label{api/ftp:limeade.ftp.views.account_edit}\pysiglinewithargsret{\code{limeade.ftp.views.}\bfcode{account\_edit}}{\emph{request}, \emph{*args}, \emph{**kwargs}}{}
Change an ftp accounts password.
\begin{quote}\begin{description}
\item[{Parameters}] \leavevmode\begin{itemize}
\item {} 
\textbf{request} -- the request object

\item {} 
\textbf{slug} -- the id of the ftp account

\end{itemize}

\item[{Returns}] \leavevmode
an edit form template

\end{description}\end{quote}

\end{fulllineitems}

\index{account\_list() (in module limeade.ftp.views)}

\begin{fulllineitems}
\phantomsection\label{api/ftp:limeade.ftp.views.account_list}\pysiglinewithargsret{\code{limeade.ftp.views.}\bfcode{account\_list}}{\emph{request}, \emph{*args}, \emph{**kwargs}}{}
Show a list of FTP accounts.
\begin{quote}\begin{description}
\item[{Parameters}] \leavevmode
\textbf{request} -- the request object

\item[{Returns}] \leavevmode
a list of ftp accounts

\end{description}\end{quote}

\end{fulllineitems}



\section{Cluster}
\label{api/cluster:cluster}\label{api/cluster::doc}

\subsection{Models}
\label{api/cluster:models}\label{api/cluster:module-limeade.cluster.models}\index{limeade.cluster.models (module)}
Models for limeade cluster
\index{Region (class in limeade.cluster.models)}

\begin{fulllineitems}
\phantomsection\label{api/cluster:limeade.cluster.models.Region}\pysiglinewithargsret{\strong{class }\code{limeade.cluster.models.}\bfcode{Region}}{\emph{*args}, \emph{**kwargs}}{}
Creates a cluster region.
\begin{quote}\begin{description}
\item[{Parameters}] \leavevmode
\textbf{name} -- name of the region

\end{description}\end{quote}

\end{fulllineitems}

\index{Server (class in limeade.cluster.models)}

\begin{fulllineitems}
\phantomsection\label{api/cluster:limeade.cluster.models.Server}\pysiglinewithargsret{\strong{class }\code{limeade.cluster.models.}\bfcode{Server}}{\emph{*args}, \emph{**kwargs}}{}
Creates a cluster server.
\begin{quote}\begin{description}
\item[{Parameters}] \leavevmode\begin{itemize}
\item {} 
\textbf{hostname} -- name of the region

\item {} 
\textbf{ip} -- the ip of the server

\item {} 
\textbf{region} -- foreign key to cluster region

\item {} 
\textbf{services} -- foreign key to cluster service

\item {} 
\textbf{enabled} -- indicates if this server is enabled

\end{itemize}

\end{description}\end{quote}

\end{fulllineitems}

\index{Service (class in limeade.cluster.models)}

\begin{fulllineitems}
\phantomsection\label{api/cluster:limeade.cluster.models.Service}\pysiglinewithargsret{\strong{class }\code{limeade.cluster.models.}\bfcode{Service}}{\emph{*args}, \emph{**kwargs}}{}
Creates a cluster service.
\begin{quote}\begin{description}
\item[{Parameters}] \leavevmode
\textbf{name} -- name of the service

\end{description}\end{quote}

\end{fulllineitems}



\subsection{Views}
\label{api/cluster:module-limeade.cluster.views}\label{api/cluster:views}\index{limeade.cluster.views (module)}
Views for limeade cluster
\index{region\_add() (in module limeade.cluster.views)}

\begin{fulllineitems}
\phantomsection\label{api/cluster:limeade.cluster.views.region_add}\pysiglinewithargsret{\code{limeade.cluster.views.}\bfcode{region\_add}}{\emph{request}, \emph{*args}, \emph{**kwargs}}{}
Add a new region to the cluster.
\begin{quote}\begin{description}
\item[{Parameters}] \leavevmode
\textbf{request} -- the request object

\item[{Returns}] \leavevmode
an edit form template

\end{description}\end{quote}

\end{fulllineitems}

\index{region\_delete() (in module limeade.cluster.views)}

\begin{fulllineitems}
\phantomsection\label{api/cluster:limeade.cluster.views.region_delete}\pysiglinewithargsret{\code{limeade.cluster.views.}\bfcode{region\_delete}}{\emph{request}, \emph{*args}, \emph{**kwargs}}{}
Remove a region from the cluster
\begin{quote}\begin{description}
\item[{Parameters}] \leavevmode\begin{itemize}
\item {} 
\textbf{request} -- the request object

\item {} 
\textbf{slug} -- the id of the region

\end{itemize}

\item[{Returns}] \leavevmode
redirect to list of regions

\end{description}\end{quote}

\end{fulllineitems}

\index{region\_edit() (in module limeade.cluster.views)}

\begin{fulllineitems}
\phantomsection\label{api/cluster:limeade.cluster.views.region_edit}\pysiglinewithargsret{\code{limeade.cluster.views.}\bfcode{region\_edit}}{\emph{request}, \emph{*args}, \emph{**kwargs}}{}
Edit a region.
\begin{quote}\begin{description}
\item[{Parameters}] \leavevmode\begin{itemize}
\item {} 
\textbf{request} -- the request object

\item {} 
\textbf{slug} -- the id of the region

\end{itemize}

\item[{Returns}] \leavevmode
an edit form template

\end{description}\end{quote}

\end{fulllineitems}

\index{region\_list() (in module limeade.cluster.views)}

\begin{fulllineitems}
\phantomsection\label{api/cluster:limeade.cluster.views.region_list}\pysiglinewithargsret{\code{limeade.cluster.views.}\bfcode{region\_list}}{\emph{request}, \emph{*args}, \emph{**kwargs}}{}
Show a list of regions.
\begin{quote}\begin{description}
\item[{Parameters}] \leavevmode
\textbf{request} -- the request object

\item[{Returns}] \leavevmode
a list of regions

\end{description}\end{quote}

\end{fulllineitems}

\index{server\_add() (in module limeade.cluster.views)}

\begin{fulllineitems}
\phantomsection\label{api/cluster:limeade.cluster.views.server_add}\pysiglinewithargsret{\code{limeade.cluster.views.}\bfcode{server\_add}}{\emph{request}, \emph{*args}, \emph{**kwargs}}{}
Add a new server to the cluster.
\begin{quote}\begin{description}
\item[{Parameters}] \leavevmode
\textbf{request} -- the request object

\item[{Returns}] \leavevmode
an edit form template

\end{description}\end{quote}

\end{fulllineitems}

\index{server\_delete() (in module limeade.cluster.views)}

\begin{fulllineitems}
\phantomsection\label{api/cluster:limeade.cluster.views.server_delete}\pysiglinewithargsret{\code{limeade.cluster.views.}\bfcode{server\_delete}}{\emph{request}, \emph{*args}, \emph{**kwargs}}{}
Remove a server from the cluster.
\begin{quote}\begin{description}
\item[{Parameters}] \leavevmode\begin{itemize}
\item {} 
\textbf{request} -- the request object

\item {} 
\textbf{slug} -- the id of the server

\end{itemize}

\item[{Returns}] \leavevmode
redirect to server list

\end{description}\end{quote}

\end{fulllineitems}

\index{server\_disable() (in module limeade.cluster.views)}

\begin{fulllineitems}
\phantomsection\label{api/cluster:limeade.cluster.views.server_disable}\pysiglinewithargsret{\code{limeade.cluster.views.}\bfcode{server\_disable}}{\emph{request}, \emph{*args}, \emph{**kwargs}}{}
Put a server into maintenance.
\begin{quote}\begin{description}
\item[{Parameters}] \leavevmode\begin{itemize}
\item {} 
\textbf{request} -- the request object

\item {} 
\textbf{slug} -- the id of the server

\end{itemize}

\item[{Returns}] \leavevmode
redirect to server list

\end{description}\end{quote}

\end{fulllineitems}

\index{server\_edit() (in module limeade.cluster.views)}

\begin{fulllineitems}
\phantomsection\label{api/cluster:limeade.cluster.views.server_edit}\pysiglinewithargsret{\code{limeade.cluster.views.}\bfcode{server\_edit}}{\emph{request}, \emph{*args}, \emph{**kwargs}}{}
Edit a server.
\begin{quote}\begin{description}
\item[{Parameters}] \leavevmode\begin{itemize}
\item {} 
\textbf{request} -- the request object

\item {} 
\textbf{slug} -- the id of the server

\end{itemize}

\item[{Returns}] \leavevmode
an edit form template

\end{description}\end{quote}

\end{fulllineitems}

\index{server\_enable() (in module limeade.cluster.views)}

\begin{fulllineitems}
\phantomsection\label{api/cluster:limeade.cluster.views.server_enable}\pysiglinewithargsret{\code{limeade.cluster.views.}\bfcode{server\_enable}}{\emph{request}, \emph{*args}, \emph{**kwargs}}{}
Enable a server after a maintenance is over.
\begin{quote}\begin{description}
\item[{Parameters}] \leavevmode\begin{itemize}
\item {} 
\textbf{request} -- the request object

\item {} 
\textbf{slug} -- the id of the server

\end{itemize}

\item[{Returns}] \leavevmode
redirect to server list

\end{description}\end{quote}

\end{fulllineitems}

\index{server\_list() (in module limeade.cluster.views)}

\begin{fulllineitems}
\phantomsection\label{api/cluster:limeade.cluster.views.server_list}\pysiglinewithargsret{\code{limeade.cluster.views.}\bfcode{server\_list}}{\emph{request}, \emph{*args}, \emph{**kwargs}}{}
Show a list of servers.
\begin{quote}\begin{description}
\item[{Parameters}] \leavevmode
\textbf{request} -- the request object

\item[{Returns}] \leavevmode
a list of servers

\end{description}\end{quote}

\end{fulllineitems}



\section{Cloud}
\label{api/cloud::doc}\label{api/cloud:cloud}

\subsection{Models}
\label{api/cloud:models}\label{api/cloud:module-limeade.cloud.models}\index{limeade.cloud.models (module)}\index{Instance (class in limeade.cloud.models)}

\begin{fulllineitems}
\phantomsection\label{api/cloud:limeade.cloud.models.Instance}\pysiglinewithargsret{\strong{class }\code{limeade.cloud.models.}\bfcode{Instance}}{\emph{*args}, \emph{**kwargs}}{}
Creates a cloud instance
\begin{quote}\begin{description}
\item[{Parameters}] \leavevmode\begin{itemize}
\item {} 
\textbf{hostname} -- the hostname of the instance

\item {} 
\textbf{sshkeys} -- foreign key to the ssh keys

\item {} 
\textbf{owner} -- the owner of the instance

\item {} 
\textbf{domain} -- domain of the instance

\item {} 
\textbf{node} -- foreign key to the cloud node

\item {} 
\textbf{active} -- indicates if the instance is active

\item {} 
\textbf{mac\_addr} -- the mac address of the instance

\end{itemize}

\end{description}\end{quote}
\index{generate\_mac\_addr() (limeade.cloud.models.Instance method)}

\begin{fulllineitems}
\phantomsection\label{api/cloud:limeade.cloud.models.Instance.generate_mac_addr}\pysiglinewithargsret{\bfcode{generate\_mac\_addr}}{}{}
Creates a mac address.

\end{fulllineitems}

\index{save() (limeade.cloud.models.Instance method)}

\begin{fulllineitems}
\phantomsection\label{api/cloud:limeade.cloud.models.Instance.save}\pysiglinewithargsret{\bfcode{save}}{\emph{**kwargs}}{}
Saves the instance.

\end{fulllineitems}


\end{fulllineitems}

\index{Limitset (class in limeade.cloud.models)}

\begin{fulllineitems}
\phantomsection\label{api/cloud:limeade.cloud.models.Limitset}\pysiglinewithargsret{\strong{class }\code{limeade.cloud.models.}\bfcode{Limitset}}{\emph{*args}, \emph{**kwargs}}{}
Maxmimum limit for an cloud instance.
\begin{quote}\begin{description}
\item[{Parameters}] \leavevmode\begin{itemize}
\item {} 
\textbf{product} -- foreign key to the product

\item {} 
\textbf{cpu\_cores} -- max number of cpu cores

\item {} 
\textbf{memory} -- max number of memory

\item {} 
\textbf{storage} -- max number of storage

\end{itemize}

\end{description}\end{quote}
\index{utilization() (limeade.cloud.models.Limitset static method)}

\begin{fulllineitems}
\phantomsection\label{api/cloud:limeade.cloud.models.Limitset.utilization}\pysiglinewithargsret{\strong{static }\bfcode{utilization}}{\emph{user}, \emph{ressource}}{}
Returns the correct ressource.

\end{fulllineitems}


\end{fulllineitems}

\index{Node (class in limeade.cloud.models)}

\begin{fulllineitems}
\phantomsection\label{api/cloud:limeade.cloud.models.Node}\pysiglinewithargsret{\strong{class }\code{limeade.cloud.models.}\bfcode{Node}}{\emph{*args}, \emph{**kwargs}}{}
Creates a new node in the cloud.
\begin{quote}\begin{description}
\item[{Parameters}] \leavevmode\begin{itemize}
\item {} 
\textbf{name} -- the name of the node

\item {} 
\textbf{uri} -- the unique resource identifier of the node

\end{itemize}

\end{description}\end{quote}

\end{fulllineitems}

\index{SSHKey (class in limeade.cloud.models)}

\begin{fulllineitems}
\phantomsection\label{api/cloud:limeade.cloud.models.SSHKey}\pysiglinewithargsret{\strong{class }\code{limeade.cloud.models.}\bfcode{SSHKey}}{\emph{*args}, \emph{**kwargs}}{}
Creates a SSH Key for the cloud instance.
\begin{quote}\begin{description}
\item[{Parameters}] \leavevmode\begin{itemize}
\item {} 
\textbf{comment} -- comment for this key

\item {} 
\textbf{key} -- the ssh key

\item {} 
\textbf{owner} -- the owner of this key

\end{itemize}

\end{description}\end{quote}

\end{fulllineitems}



\subsection{Views}
\label{api/cloud:views}\label{api/cloud:module-limeade.cloud.views.api}\index{limeade.cloud.views.api (module)}
Views for limeade cloud API calls
\index{instance\_activate() (in module limeade.cloud.views.api)}

\begin{fulllineitems}
\phantomsection\label{api/cloud:limeade.cloud.views.api.instance_activate}\pysiglinewithargsret{\code{limeade.cloud.views.api.}\bfcode{instance\_activate}}{\emph{request}}{}
Called when an instance is done provisioning.
\begin{quote}\begin{description}
\item[{Parameters}] \leavevmode
\textbf{request} -- the request object

\end{description}\end{quote}

Requires two GET Parameters:
\begin{itemize}
\item {} 
site\_api\_key: for security reasons

\item {} 
instance: ID of the instance to activate

\end{itemize}
\begin{quote}\begin{description}
\item[{Returns}] \leavevmode
a http response with data

\end{description}\end{quote}

\end{fulllineitems}

\index{instance\_info() (in module limeade.cloud.views.api)}

\begin{fulllineitems}
\phantomsection\label{api/cloud:limeade.cloud.views.api.instance_info}\pysiglinewithargsret{\code{limeade.cloud.views.api.}\bfcode{instance\_info}}{\emph{request}, \emph{slug}}{}
Returns information an instance can use to configure itself.
\begin{quote}\begin{description}
\item[{Parameters}] \leavevmode\begin{itemize}
\item {} 
\textbf{request} -- the request object

\item {} 
\textbf{slug} -- the mac address of the instance

\end{itemize}

\end{description}\end{quote}

It includes:
\begin{itemize}
\item {} 
the hostname

\item {} 
SSH Keys

\end{itemize}
\begin{quote}\begin{description}
\item[{Returns}] \leavevmode
a http response with data

\end{description}\end{quote}

\end{fulllineitems}

\phantomsection\label{api/cloud:module-limeade.cloud.views.instance}\index{limeade.cloud.views.instance (module)}
Views for limeade cloud Instances
\index{instance\_add() (in module limeade.cloud.views.instance)}

\begin{fulllineitems}
\phantomsection\label{api/cloud:limeade.cloud.views.instance.instance_add}\pysiglinewithargsret{\code{limeade.cloud.views.instance.}\bfcode{instance\_add}}{\emph{request}, \emph{*args}, \emph{**kwargs}}{}
View for adding a new instance.
\begin{quote}\begin{description}
\item[{Parameters}] \leavevmode
\textbf{request} -- the request object

\item[{Returns}] \leavevmode
an edit form template

\end{description}\end{quote}

\end{fulllineitems}

\index{instance\_delete() (in module limeade.cloud.views.instance)}

\begin{fulllineitems}
\phantomsection\label{api/cloud:limeade.cloud.views.instance.instance_delete}\pysiglinewithargsret{\code{limeade.cloud.views.instance.}\bfcode{instance\_delete}}{\emph{request}, \emph{*args}, \emph{**kwargs}}{}
Delete a instance.
\begin{quote}\begin{description}
\item[{Parameters}] \leavevmode\begin{itemize}
\item {} 
\textbf{request} -- the request object

\item {} 
\textbf{slug} -- the id of the instance

\end{itemize}

\item[{Returns}] \leavevmode
redirect to a list of cloud instances

\end{description}\end{quote}

\end{fulllineitems}

\index{instance\_list() (in module limeade.cloud.views.instance)}

\begin{fulllineitems}
\phantomsection\label{api/cloud:limeade.cloud.views.instance.instance_list}\pysiglinewithargsret{\code{limeade.cloud.views.instance.}\bfcode{instance\_list}}{\emph{request}, \emph{*args}, \emph{**kwargs}}{}
Show a list of the current users instances.
\begin{quote}\begin{description}
\item[{Parameters}] \leavevmode
\textbf{request} -- the request object

\item[{Returns}] \leavevmode
list of instances

\end{description}\end{quote}

\end{fulllineitems}

\index{instance\_restart() (in module limeade.cloud.views.instance)}

\begin{fulllineitems}
\phantomsection\label{api/cloud:limeade.cloud.views.instance.instance_restart}\pysiglinewithargsret{\code{limeade.cloud.views.instance.}\bfcode{instance\_restart}}{\emph{request}, \emph{*args}, \emph{**kwargs}}{}
Restart a instance.
\begin{quote}\begin{description}
\item[{Parameters}] \leavevmode\begin{itemize}
\item {} 
\textbf{request} -- the request object

\item {} 
\textbf{slug} -- the id of the instance

\end{itemize}

\item[{Returns}] \leavevmode
redirect to a list of cloud instances

\end{description}\end{quote}

\end{fulllineitems}

\index{instance\_start() (in module limeade.cloud.views.instance)}

\begin{fulllineitems}
\phantomsection\label{api/cloud:limeade.cloud.views.instance.instance_start}\pysiglinewithargsret{\code{limeade.cloud.views.instance.}\bfcode{instance\_start}}{\emph{request}, \emph{*args}, \emph{**kwargs}}{}
Start a instance.
\begin{quote}\begin{description}
\item[{Parameters}] \leavevmode\begin{itemize}
\item {} 
\textbf{request} -- the request object

\item {} 
\textbf{slug} -- the id of the instance

\end{itemize}

\item[{Returns}] \leavevmode
redirect to a list of cloud instances

\end{description}\end{quote}

\end{fulllineitems}

\index{instance\_stop() (in module limeade.cloud.views.instance)}

\begin{fulllineitems}
\phantomsection\label{api/cloud:limeade.cloud.views.instance.instance_stop}\pysiglinewithargsret{\code{limeade.cloud.views.instance.}\bfcode{instance\_stop}}{\emph{request}, \emph{*args}, \emph{**kwargs}}{}
Stop a instance.
\begin{quote}\begin{description}
\item[{Parameters}] \leavevmode\begin{itemize}
\item {} 
\textbf{request} -- the request object

\item {} 
\textbf{slug} -- the id of the instance

\end{itemize}

\item[{Returns}] \leavevmode
redirect to a list of cloud instances

\end{description}\end{quote}

\end{fulllineitems}

\phantomsection\label{api/cloud:module-limeade.cloud.views.ssh}\index{limeade.cloud.views.ssh (module)}
Views for limeade cloud SSH Keys
\index{sshkey\_add() (in module limeade.cloud.views.ssh)}

\begin{fulllineitems}
\phantomsection\label{api/cloud:limeade.cloud.views.ssh.sshkey_add}\pysiglinewithargsret{\code{limeade.cloud.views.ssh.}\bfcode{sshkey\_add}}{\emph{request}, \emph{*args}, \emph{**kwargs}}{}
Form to add a new SSH Key.
\begin{quote}\begin{description}
\item[{Parameters}] \leavevmode
\textbf{request} -- the request object

\item[{Returns}] \leavevmode
an edit form template

\end{description}\end{quote}

\end{fulllineitems}

\index{sshkey\_delete() (in module limeade.cloud.views.ssh)}

\begin{fulllineitems}
\phantomsection\label{api/cloud:limeade.cloud.views.ssh.sshkey_delete}\pysiglinewithargsret{\code{limeade.cloud.views.ssh.}\bfcode{sshkey\_delete}}{\emph{request}, \emph{*args}, \emph{**kwargs}}{}
Delete a SSH Key.
\begin{quote}\begin{description}
\item[{Parameters}] \leavevmode\begin{itemize}
\item {} 
\textbf{request} -- the request object

\item {} 
\textbf{slug} -- the id of the ssh key

\end{itemize}

\item[{Returns}] \leavevmode
redirect to a list of ssh keys

\end{description}\end{quote}

\end{fulllineitems}

\index{sshkey\_list() (in module limeade.cloud.views.ssh)}

\begin{fulllineitems}
\phantomsection\label{api/cloud:limeade.cloud.views.ssh.sshkey_list}\pysiglinewithargsret{\code{limeade.cloud.views.ssh.}\bfcode{sshkey\_list}}{\emph{request}, \emph{*args}, \emph{**kwargs}}{}
Show a list of the Users SSH Keys.
\begin{quote}\begin{description}
\item[{Parameters}] \leavevmode
\textbf{request} -- the request object

\item[{Returns}] \leavevmode
a list of ssh keys

\end{description}\end{quote}

\end{fulllineitems}

\phantomsection\label{api/cloud:module-limeade.cloud.views.vnc}\index{limeade.cloud.views.vnc (module)}
Views for limeade cloud VNC
\index{instance\_vnc() (in module limeade.cloud.views.vnc)}

\begin{fulllineitems}
\phantomsection\label{api/cloud:limeade.cloud.views.vnc.instance_vnc}\pysiglinewithargsret{\code{limeade.cloud.views.vnc.}\bfcode{instance\_vnc}}{\emph{request}, \emph{*args}, \emph{**kwargs}}{}
Lets view the authenticated user a VNC console.
\begin{quote}\begin{description}
\item[{Parameters}] \leavevmode\begin{itemize}
\item {} 
\textbf{request} -- the request object

\item {} 
\textbf{slug} -- the virtual machine id

\end{itemize}

\item[{Returns}] \leavevmode
the vnc template

\item[{Raises }] \leavevmode
DoesNotExist

\end{description}\end{quote}

\end{fulllineitems}

\index{instance\_vnc\_auth() (in module limeade.cloud.views.vnc)}

\begin{fulllineitems}
\phantomsection\label{api/cloud:limeade.cloud.views.vnc.instance_vnc_auth}\pysiglinewithargsret{\code{limeade.cloud.views.vnc.}\bfcode{instance\_vnc\_auth}}{\emph{request}, \emph{slug}, \emph{token}}{}
API call for websocket to tcp proxy to get the settings.
\begin{quote}\begin{description}
\item[{Parameters}] \leavevmode\begin{itemize}
\item {} 
\textbf{request} -- the request object

\item {} 
\textbf{slug} -- the virtual machine id

\item {} 
\textbf{token} -- the session id from the requesting user

\end{itemize}

\item[{Returns}] \leavevmode
dictionary containing proxy settings

\item[{Raises }] \leavevmode
ObjectDoesNotExist

\end{description}\end{quote}

\end{fulllineitems}



\chapter{Index und Tabellen}
\label{index:index-und-tabellen}
Benötigten Informationen nicht gefunden? Versuche es im Index oder probiere die
Suchfunktion:
\begin{itemize}
\item {} 
\emph{genindex}

\item {} 
\emph{search}

\end{itemize}


\renewcommand{\indexname}{Python Module Index}
\begin{theindex}
\def\bigletter#1{{\Large\sffamily#1}\nopagebreak\vspace{1mm}}
\bigletter{l}
\item {\texttt{limeade.cloud.models}}, \pageref{api/cloud:module-limeade.cloud.models}
\item {\texttt{limeade.cloud.views.api}}, \pageref{api/cloud:module-limeade.cloud.views.api}
\item {\texttt{limeade.cloud.views.instance}}, \pageref{api/cloud:module-limeade.cloud.views.instance}
\item {\texttt{limeade.cloud.views.ssh}}, \pageref{api/cloud:module-limeade.cloud.views.ssh}
\item {\texttt{limeade.cloud.views.vnc}}, \pageref{api/cloud:module-limeade.cloud.views.vnc}
\item {\texttt{limeade.cluster.models}}, \pageref{api/cluster:module-limeade.cluster.models}
\item {\texttt{limeade.cluster.views}}, \pageref{api/cluster:module-limeade.cluster.views}
\item {\texttt{limeade.ftp.models}}, \pageref{api/ftp:module-limeade.ftp.models}
\item {\texttt{limeade.ftp.views}}, \pageref{api/ftp:module-limeade.ftp.views}
\item {\texttt{limeade.mail.models}}, \pageref{api/mail:module-limeade.mail.models}
\item {\texttt{limeade.mail.views}}, \pageref{api/mail:module-limeade.mail.views}
\item {\texttt{limeade.mysql.models}}, \pageref{api/mysql:module-limeade.mysql.models}
\item {\texttt{limeade.mysql.views}}, \pageref{api/mysql:module-limeade.mysql.views}
\item {\texttt{limeade.system.models}}, \pageref{api/system:module-limeade.system.models}
\item {\texttt{limeade.system.views}}, \pageref{api/system:module-limeade.system.views}
\item {\texttt{limeade.web.models}}, \pageref{api/web:module-limeade.web.models}
\item {\texttt{limeade.web.views}}, \pageref{api/web:module-limeade.web.views}
\end{theindex}

\renewcommand{\indexname}{Index}
\printindex
\end{document}
